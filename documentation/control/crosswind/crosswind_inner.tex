\documentclass[11pt]{amsart}
\usepackage{geometry}
\geometry{letterpaper}
\usepackage{graphicx}
\usepackage{amssymb}
\usepackage{amsmath}
\usepackage{siunitx}
\usepackage{tikz}
\usepackage{lscape}
\usetikzlibrary{dsp,fit,positioning}

\makeatletter
\dspdeclareoperator{dspvoidshapeadder}{
  % Coordinate offset for the plus
  \pgfutil@tempdima=\radius
  \pgfutil@tempdima=0.55\pgfutil@tempdima
  \pgfusepathqstroke
}
\tikzset{
  vdspadder/.style={
    shape=dspvoidshapeadder,
    line cap=rect,
    line join=rect,
    line width=\dspblocklinewidth,
    minimum size=\dspoperatordiameter,
    label={185:$+$},
    label={265:$-$}
  },
  vadspadder/.style={
    shape=dspvoidshapeadder,
    line cap=rect,
    line join=rect,
    line width=\dspblocklinewidth,
    minimum size=\dspoperatordiameter,
    label=below right:$-$,
    label=above right:$+$
  },
  vkdspadder/.style={
    shape=dspvoidshapeadder,
    line cap=rect,
    line join=rect,
    line width=\dspblocklinewidth,
    minimum size=\dspoperatordiameter,
    label=below right:$-$,
    label=below left:$+$
  },
  vjdspadder/.style={
    shape=dspvoidshapeadder,
    line cap=rect,
    line join=rect,
    line width=\dspblocklinewidth,
    minimum size=\dspoperatordiameter,
    label=above right:$+$,
    label=below right:$-$
  },
  edspadder/.style={
    shape=dspvoidshapeadder,
    line cap=rect,
    line join=rect,
    line width=\dspblocklinewidth,
    minimum size=\dspoperatordiameter
  }
}
\makeatother


\tikzset{dspblock/.style={inner xsep = 5pt, inner ysep = 5pt}}
\tikzset{dspblock2/.style={inner xsep = 5pt, inner ysep = 10pt}}
\tikzset{dspblock3/.style={inner xsep = 5pt, inner ysep = 20pt}}


\newcommand{\qhat}{\hat{q}}
\newcommand{\cbar}{\bar{c}}
\newcommand{\qbar}{\bar{q}}
\newcommand{\cmd}{\mathrm{cmd}}
\newcommand{\ff}{\mathrm{ff}}
\newcommand{\eff}{\mathrm{eff}}
\newcommand{\app}{\mathrm{app}}
\newcommand{\wind}{\mathrm{wind}}
\newcommand{\kite}{\mathrm{kite}}
\newcommand{\nom}{\mathrm{nom}}
\newcommand{\err}{\mathrm{err}}
\newcommand{\aero}{\mathrm{aero}}
\newcommand{\geom}{\mathrm{geom}}

\begin{document}

%% \section{Inner loops of the crosswind controller}

%% We assume that the longitudinal dynamics can be separated into the
%% airspeed dynamics and the angle-of-attack dynamics.  This

%% \subsection{Airspeed loop}

%% For control law design, the airspeed dynamics are approximated by the
%% inertial speed dynamics.  The model is based around linearizing the
%% speed dynamics about the steady-state speed, which is the lift-to-drag
%% ratio multiplied by the wind speed:
%% \begin{equation}
%%   v = \frac{C_L}{C_D} v_{\wind}
%% \end{equation}
%% There is an effective restoring force for any perturbations away from
%% this value described by $k_{\eff}$ in the equations below.  This
%% effective restoring force depends on wind speed; however, in the
%% current control law design a representative wind speed is chosen.
%% \begin{equation}
%% G_{\mathrm{airspeed}} = \frac{1}{m_{\eff} s + k_{\eff}}
%% \end{equation}
%% \begin{equation}
%% k_{\eff} = \frac{1}{2} \rho A C_{L_0} v_{\wind}
%% \end{equation}


%% \begin{center}
%%   \begin{table}[h]
%%     \label{tbl:airspeed_gains}
%%     \caption{Gains for the airspeed loop.}
%%     \begin{tabular}{lllllll}
%%       \hline
%%       \hline
%%       Variable & Value & Units\\
%%       \hline
%%       $k_p$ & $3.21 \times 10^3$ & N / (m/s) \\
%%       $k_i$ & $1.55 \times 10^3$ & N $\cdot$ s / (m/s) \\
%%       $k_d$ & 0.0                & N / (m/s) / s \\
%%       \hline
%%       \hline
%%     \end{tabular}
%%   \end{table}
%% \end{center}


%% \begin{center}
%% \begin{figure}[h]
%% \begin{tikzpicture}
%%   \matrix [row sep=4mm, column sep=8mm] {
%%     \node[coordinate] (n00) {}; &
%%     \node[dspnodeopen, dsp/label=left] (n01)
%%          {$T_{\ff} = m \vec{g}_b \cdot \hat{x}_b$}; \\
%%     \node[coordinate] (n02) {}; &[-4mm]
%%     \node[coordinate] (n03) {}; &
%%     \node[coordinate] (n04) {}; &
%%     \node[coordinate] (n05) {}; &
%%     \node[coordinate] (n06) {}; &
%%     \node[coordinate] (n07) {}; & \\
%%     \node[dspnodeopen, dsp/label=above] (n10) {$v_{\mathrm{cmd}}$}; &[-4mm]
%%     \node[vdspadder] (n11) {}; &[-4mm]
%%     \node[dspfilter] (n12) {$C_{\mathrm{airspeed}}$}; &[-4mm]
%%     \node[dspadder] (n13) {}; &
%%     \node[dspfilter] (n14) {$G_{\mathrm{motors}}$}; &
%%     \node[dspfilter] (n15) {$G_{\mathrm{airspeed}}$}; &
%%     \node[dspnodefull] (n16) {}; &
%%     \node[dspnodeopen, dsp/label=above] (n17) {$v$}; \\
%%     \node[coordinate] (n20) {}; &
%%     \node[coordinate] (n21) {}; &
%%     \node[coordinate] (n22) {}; &[-4mm]
%%     \node[coordinate] (n23) {}; &
%%     \node[coordinate] (n24) {}; &
%%     \node[coordinate] (n25) {}; &
%%     \node[coordinate] (n26) {}; &
%%     \node[coordinate] (n27) {}; & \\
%%   };

%%   \draw[dspconn] (n01) -| (n13);
%%   \draw[dspconn] (n10) -- (n11);
%%   \draw[dspconn] (n11) -- (n12) -- (n13);
%%   \draw[dspconn] (n13) -- node[above] {$T_{\mathrm{cmd}}$}
%%                  (n14) -- node[above] {$T$} (n15) -- (n16) -- (n17);
%%   \draw[dspconn] (n16) |- (n26) -| (n11);
%% \end{tikzpicture}
%% \end{figure}
%% \end{center}





%% \section{Longitudinal loop}

%% \begin{equation}
%% \mathbf{u} = \left[\;\;\delta_e,\; \delta_f \right\;\;]^T
%% \end{equation}

%% \begin{equation}
%% \alpha_{\cmd},\; q_{\cmd},\; \Delta C_L_{\cmd}
%% \end{equation}




\section{Lateral control}

The lateral control loop controls the tether roll angle, sideslip
angle, roll rate, and yaw rate using the aileron and rudder.  The
aileron and rudder commands are chosen with state feedback where the
state is augmented with integrals of the tether roll error and the
sideslip error:
%
\begin{equation}
\mathbf{u} = \mathbf{K} (\mathbf{x}'_{\cmd} - \mathbf{x}')
\end{equation}
%
where
%
\begin{equation}
\mathbf{u} = [\delta_a, \delta_r]^T
\end{equation}
%
\begin{equation}
\mathbf{x}' = \left[ \phi_t, \beta, p, r, \smallint {\phi_t}_{\err}\, dt,
                    \smallint \beta_{\err}\, dt \right]^T
\end{equation}

\begin{center}
\begin{figure}[h]
\begin{tikzpicture}
  \matrix [row sep=4mm, column sep=10mm] {
    \node[dspnodeopen, dsp/label=above] (n00) {$\mathbf{x}_{\cmd}$}; &
    \node[dspnodefull] (n01) {}; &[-4mm]
    \node[coordinate] (n02) {}; &[-8mm]
    \node[vkdspadder] (n03) {}; &[-4mm]
    \node[coordinate] (n04) {}; &
    \node[coordinate] (n05) {}; &
    \node[coordinate] (n06) {}; &
    \node[coordinate] (n07) {}; & \\[-4mm]
%
    \node[coordinate] (n10) {}; &
    \node[vjdspadder] (n11) {}; &[-4mm]
    \node[dspsquare] (n12) {$\int$}; &[-8mm]
    \node[coordinate] (n13) {}; &[-4mm]
    \node[dspfilter, minimum height=1.5cm] (n14) {$K$}; &
    \node[dspfilter] (n15) {$G_{\mathrm{lateral}}$}; &
    \node[dspnodefull] (n16) {}; &
    \node[dspnodeopen, dsp/label=above] (n17) {$\phi_t,\; \beta$}; \\
%
    \node[coordinate] (n20) {}; &
    \node[coordinate] (n21) {}; &[-4mm]
    \node[coordinate] (n22) {}; &[-8mm]
    \node[coordinate] (n23) {}; &[-4mm]
    \node[coordinate] (n24) {}; &
    \node[coordinate] (n25) {}; &
    \node[coordinate] (n26) {}; &
    \node[coordinate] (n27) {}; & \\
  };

  \draw[dspconn] (n00) -- (n03) {};
  \draw[dspconn] (n01) -- node[left]{$\phi_t_{\cmd},\; \beta_{\cmd}$}
                 (n11);
  \draw[dspconn] (n15) |-
                 (n05) -- node[above] {$\mathbf{x} = [\phi_t, \beta, p, r]^T$}
                 (n03);
  \draw[dspconn] (n03) |- (n14.160);
  \draw[dspconn] (n11) -- (n12) -- (n13) -- (n14);
  \draw[dspconn] (n14) -- node[above] {$\delta_a, \delta_r$}
                 (n15) -- (n16) -- (n17);
  \draw[dspconn] (n16) |- (n26) -| (n11);
\end{tikzpicture}
\end{figure}
\end{center}


The gain matrix is scheduled based on airspeed using linear
intepolation between the following matrices at airspeeds 30 m/s, 60
m/s, and 90 m/s.

\begin{equation}
\mathbf{K}_{\min} = \left[
\begin{array}{cccccc}
  1.946 & -1.974 & -0.422 & 0.419 & -0.727 & 0.449 \\
  0.199 & 1.281 & 0.055 & -0.722 & -0.897 & -0.363 \\
\end{array}
\right]
\end{equation}

\begin{equation}
\mathbf{K}_{\nom} = \left[
\begin{array}{cccccc}
  1.022 & -0.966 & -0.159 & 0.11 & -0.915 & 0.352 \\
  0.186 & 1.525 & 0.048 & -0.458 & -0.705 & -0.457 \\
\end{array}
\right]
\end{equation}

\begin{equation}
\mathbf{K}_{\max} = \left[
\begin{array}{cccccc}
  0.909 & -0.771 & -0.121 & 0.082 & -0.985 & 0.302 \\
  0.047 & 1.56 & 0.02 & -0.373 & -0.603 & -0.492 \\
\end{array}
\right]
\end{equation}

See
\texttt{go/makanics/config/m600/control/crosswind\_controllers.py}
for the most up-to-date gains.

%% The typical lateral dynamics state vector consists of the roll angle,
%% y velocity, in body coordinates, and the roll rate and yaw rate.
%% However, once the tether is involved, the 

%% \begin{equation}
%% \mathbf{x} = \left[ y,\; \phi,\; \psi,\; \dot{y},\; \dot{\phi},\; \dot{\psi} \right]^T
%% \end{equation}

%% \begin{equation}
%% \mathbf{u} = \left[\;\;\delta_a,\; \delta_r,\; \tau_z \;\;\right]^T
%% \end{equation}

%% \begin{equation}
%% \mathbf{y} = \left[\phi_t,\; \beta,\; p,\; r \right]^T
%% \end{equation}

%% \begin{equation}
%% \mathbf{y} = \left[\phi_t_{\cmd},\; \beta_{\cmd},\; p_{\cmd},\; r_{\cmd} \right]^T
%% \end{equation}

%% \begin{equation}
%% \mathbf{M} = \left[
%% \begin{array}{ccc}
%% m & 0 & 0 \\
%% 0 & I_{xx} & I_{xz} \\
%% 0 & I_{zx} & I_{zz} \\
%% \end{array}
%% \right]
%% \end{equation}

%% \begin{equation}
%% \mathbf{A}_{\mathrm{lateral}} = \qbar A
%% \left[
%% \begin{array}{cc}
%% \mathbf{I} & \mathbf{0} \\
%% \mathbf{0} & \mathbf{M}^{-1}
%% \end{array}
%% \right]
%% \cdot
%% \left[
%% \begin{array}{cccccc}
%% 0 & 0 & 0 & 1 & 0 & 0 \\
%% 0 & 0 & 0 & 0 & 1 & 0 \\
%% 0 & 0 & 0 & 0 & 0 & 1 \\
%% C_L/l_t & C_L & \left(C_{Y_{\beta}} - C_L x_b/l_t \right) &
%% C_{Y_{\beta}}/v & C_{Y_{\hat{p}}} \tau & C_{Y_{\hat{r}}} \tau \\
%% %
%% C_L z_b/l_t & -C_L z_b & \left(-C_{l_{\beta}} b + C_L z_b x_b / l_t\right) &
%% C_{l_{\beta}} b/v & C_{l_{\hat{p}}} b \tau & C_{l_{\hat{r}}} b \tau \\
%% %
%% -C_L x_b/l_t & C_L x_b & \left(-C_{n_{\beta}} b - C_L x_b^2/l_t\right) &
%% C_{n_{\beta}} b/v & C_{n_{\hat{p}}} b \tau & C_{n_{\hat{r}}} b \tau \\
%% \end{array}
%% \right]
%% \end{equation}

%% \begin{equation}
%% \mathbf{B}_{\mathrm{lateral}} = \qbar A
%% \left[
%% \begin{array}{cc}
%% \mathbf{I} & \mathbf{0} \\
%% \mathbf{0} & \mathbf{M}^{-1}
%% \end{array}
%% \right]
%% \cdot
%% \left[
%% \begin{array}{cc}
%% 0 & 0 \\
%% 0 & 0 \\
%% 0 & 0 \\
%% C_{Y_{\delta_a}} & C_{Y_{\delta_r}} \\
%% C_{l_{\delta_a}} b & C_{l_{\delta_r}} b \\
%% C_{n_{\delta_a}} b & C_{n_{\delta_r}} b \\
%% \end{array}
%% \right]
%% \end{equation}

%% \begin{equation}
%% \mathbf{C}_{\mathrm{lateral}} =
%% \left[
%% \begin{array}{cccccc}
%% 1/l_t & -1 &  0 & 0   & 0 & 0 \\
%% 0     & 0  & -1 & 1/v & 0 & 0\\
%% \end{array}
%% \right]
%% \end{equation}

%% We augment the 

\end{document}
