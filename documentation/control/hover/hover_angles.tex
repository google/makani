\documentclass{article} %[11pt]{amsart}
\usepackage{geometry}
\geometry{letterpaper}
\usepackage{graphicx}
\usepackage{amssymb}
\usepackage{amsmath}
\usepackage{siunitx}
\usepackage{tikz}
\usepackage{lscape}
\usepackage{pgfplots}
\usetikzlibrary{
  arrows,
  calc,
  decorations.markings,
  decorations.pathreplacing,
  dsp,
  fit,
  positioning
}
\makeatletter
\dspdeclareoperator{dspvoidshapeadder}{
  % Coordinate offset for the plus
  \pgfutil@tempdima=\radius
  \pgfutil@tempdima=0.55\pgfutil@tempdima
  \pgfusepathqstroke
}
\tikzset{
  vdspadder/.style={
    shape=dspvoidshapeadder,
    line cap=rect,
    line join=rect,
    line width=\dspblocklinewidth,
    minimum size=\dspoperatordiameter,
    label={185:$+$},
    label={265:$-$}
  },
  vadspadder/.style={
    shape=dspvoidshapeadder,
    line cap=rect,
    line join=rect,
    line width=\dspblocklinewidth,
    minimum size=\dspoperatordiameter,
    label=below right:$-$,
    label=above right:$+$
  },
  vkdspadder/.style={
    shape=dspvoidshapeadder,
    line cap=rect,
    line join=rect,
    line width=\dspblocklinewidth,
    minimum size=\dspoperatordiameter,
    label=below right:$-$,
    label=below left:$+$
  },
  vjdspadder/.style={
    shape=dspvoidshapeadder,
    line cap=rect,
    line join=rect,
    line width=\dspblocklinewidth,
    minimum size=\dspoperatordiameter,
    label=above right:$+$,
    label=below right:$-$
  },
  edspadder/.style={
    shape=dspvoidshapeadder,
    line cap=rect,
    line join=rect,
    line width=\dspblocklinewidth,
    minimum size=\dspoperatordiameter
  }
}
\makeatother


\tikzset{dspblock/.style={inner xsep = 5pt, inner ysep = 5pt}}
\tikzset{dspblock2/.style={inner xsep = 5pt, inner ysep = 10pt}}
\tikzset{dspblock3/.style={inner xsep = 5pt, inner ysep = 20pt}}


\newcommand{\qhat}{\hat{q}}
\newcommand{\cbar}{\bar{c}}
\newcommand{\qbar}{\bar{q}}
\newcommand{\cmd}{\mathrm{cmd}}
\newcommand{\ff}{\mathrm{ff}}
\newcommand{\eff}{\mathrm{eff}}
\newcommand{\app}{\mathrm{app}}
\newcommand{\wind}{\mathrm{wind}}
\newcommand{\kite}{\mathrm{kite}}
\newcommand{\nom}{\mathrm{nom}}
\newcommand{\aero}{\mathrm{aero}}
\newcommand{\geom}{\mathrm{geom}}

\begin{document}

\section{Describing function analysis of rate limit nonlinearity}

There are two modes for the rate limit nonlinearity: one where the
signal is only rate limited for a portion of the period and another
where the signal is rate limited for the entire duration.  For
simplicity, we focus on the latter case.  Reference \cite{pratt2000}
has a good analysis of this case, and reference \cite{ponce2003} has a
more complete analysis of both cases.

The describing function for a rate-limit, when it is completely
rate-limiting the input signal is:
\begin{eqnarray}
\gamma &=& \sin^{-1} \left( \frac{\pi}{2} \frac{R}{M \omega} \right) \\
N(M, \omega) &=& \frac{4 R}{\pi M \omega} (\sin \gamma + i \cos \gamma)
\end{eqnarray}

This may be reduced to a function of one variable $x = M \omega / R$:
\begin{eqnarray}
N(x) &=& \frac{2}{x^2} +
i \frac{4}{\pi x} \cos \sin^{-1} \left( \frac{\pi}{2 x} \right) \\
&=& \frac{2}{x^2} \left(1 + i \sqrt{\left(\frac{2 x}{\pi}\right)^2 - 1} \right)
\end{eqnarray}

\begin{equation}
-1/N(x) = -\frac{\pi^2}{8}
\left( 1 - i \sqrt{\left(\frac{2 x}{\pi}\right)^2 - 1} \right)
\end{equation}

\begin{equation}
C(s) = k_p + k_d s + k_i / s
\end{equation}

\begin{equation}
G_{\mathrm{motor}}(s) = \frac{\omega_m}{s + \omega_m}
\end{equation}

\begin{equation}
G_{\mathrm{yaw}}(s) = \frac{1}{I_{zz} s^2}
\end{equation}

\begin{equation}
L(s) = C(s) G_{\mathrm{motor}}(s) G_{\mathrm{yaw}}(s)
\end{equation}

\begin{table}
\begin{center}
\begin{tabular}{cc}
\hline
\hline
Variable & Value \\
\hline
$k_p$ & $2.58 \times 10^5$ \\
$k_i$ & $2.04 \times 10^4$ \\
$k_d$ & $1.31 \times 10^5$ \\
$I_{zz}$ & $3 \times 10^4$ \\
$\omega_m$ & $2 \pi \cdot 6$ \\
\hline
\hline
\end{tabular}
\end{center}
\end{table}

\begin{figure}[!ht]
\begin{center}
\begin{tikzpicture}
\begin{axis}[
    scale=1.3,
    xlabel=Re,
    xmin=-5,
    xmax=5,
    ylabel=Im,
    ymin=-5,
    ymax=5,
    grid=both
  ]
  \def\a{(2 / (x * x))};
  \def\b{(2 / (x * x) * sqrt((2 * x / 3.14)^2 - 1))};
  \def\m{(\a * \a + \b * \b)};
  \def\kp{2.58e5};
  \def\ki{2.04e4};
  \def\kd{1.31e5};
  \def\Izz{3e4};
  \def\omegam{(2 * 3.14 * 6)};
  \def\Ga{(-\kd * \omegam * x^2 + \ki * \omegam)};
  \def\Gb{(\kp * \omegam * x)};
  \def\Gc{(\Izz * x^4)};
  \def\Gd{(-\Izz * \omegam * x^3)};
  \def\Gm{(\Gc * \Gc + \Gd * \Gd)};
  \addplot [smooth, color=blue, mark=none, domain=0:10]
           ({-\a / \m}, {\b / \m});
  \addplot [smooth, color=green, mark=none, domain=-15:-1]
            ({(\Ga * \Gc + \Gb * \Gd) / \Gm}, {(\Gb * \Gc - \Ga * \Gd) / \Gm});
  \addplot [smooth, color=green, mark=none, domain=1:15]
            ({(\Ga * \Gc + \Gb * \Gd) / \Gm}, {(\Gb * \Gc - \Ga * \Gd) / \Gm});
\end{axis}
\end{tikzpicture}
\caption{$L(j \omega) = -1 / N(M \omega / R)$ Intersection occurs at $x = 2.44$ and
  $\omega = -2.77$ rad/s.}
\label{fig:rlocus}
\end{center}
\end{figure}


\begin{thebibliography}{1}
\bibitem{pratt2000} Pratt, Roger. ``Flight Control Systems: Practical Issues in
  Design and Implementation.'' {\it IET}. 2000.\
\bibitem{ponce2003} Ponce, E. and Roman, M. ``The describing function
  method accuracy in first order plants with rate-limited feedback.''
  {\it Proceedings of the European Control Conference}. 2003.
\end{thebibliography}

\end{document}
