\documentclass[11pt]{amsart}
\usepackage{geometry}
\geometry{letterpaper}
\usepackage{graphicx}
\usepackage{amssymb}
\usepackage{amsmath}
\usepackage{epstopdf}
\usepackage{fancyhdr}
\usepackage{tikz}
\usetikzlibrary{dsp,fit}

\newcommand\DrawAirfoil[1][]{
\begin{scope}[#1]
  % NACA 0012
  \draw plot [smooth] coordinates {
    (1.000000, 0.001260000)
    (0.9916796, 0.002421450)
    (0.9803692, 0.003981369)
    (0.9672663, 0.005761861)
    (0.9527169, 0.007706239)
    (0.9371981, 0.009743313)
    (0.9211172, 0.01181512)
    (0.9047380, 0.01388559)
    (0.8882062, 0.01593565)
    (0.8715962, 0.01795616)
    (0.8549446, 0.01994301)
    (0.8382694, 0.02189448)
    (0.8215796, 0.02380992)
    (0.8048802, 0.02568905)
    (0.7881741, 0.02753174)
    (0.7714632, 0.02933785)
    (0.7547492, 0.03110713)
    (0.7380334, 0.03283924)
    (0.7213171, 0.03453371)
    (0.7046016, 0.03618991)
    (0.6878882, 0.03780710)
    (0.6711780, 0.03938437)
    (0.6544725, 0.04092067)
    (0.6377730, 0.04241478)
    (0.6210807, 0.04386535)
    (0.6043972, 0.04527084)
    (0.5877237, 0.04662956)
    (0.5710618, 0.04793964)
    (0.5544131, 0.04919904)
    (0.5377791, 0.05040555)
    (0.5211615, 0.05155676)
    (0.5045622, 0.05265006)
    (0.4879829, 0.05368266)
    (0.4714257, 0.05465156)
    (0.4548929, 0.05555352)
    (0.4383865, 0.05638509)
    (0.4219094, 0.05714258)
    (0.4054640, 0.05782205)
    (0.3890536, 0.05841929)
    (0.3726814, 0.05892981)
    (0.3563512, 0.05934879)
    (0.3400671, 0.05967113)
    (0.3238340, 0.05989134)
    (0.3076573, 0.06000357)
    (0.2915433, 0.06000157)
    (0.2754992, 0.05987865)
    (0.2595337, 0.05962761)
    (0.2436573, 0.05924080)
    (0.2278824, 0.05871002)
    (0.2122249, 0.05802654)
    (0.1967044, 0.05718119)
    (0.1813469, 0.05616449)
    (0.1661868, 0.05496702)
    (0.1512710, 0.05358012)
    (0.1366645, 0.05199727)
    (0.1224568, 0.05021633)
    (0.1087684, 0.04824299)
    (0.09575211, 0.04609495)
    (0.08358167, 0.04380527)
    (0.07242320, 0.04142167)
    (0.06239496, 0.03899972)
    (0.05353650, 0.03659163)
    (0.04580557, 0.03423655)
    (0.03910103, 0.03195685)
    (0.03329449, 0.02976012)
    (0.02825571, 0.02764358)
    (0.02386683, 0.02559846)
    (0.02002796, 0.02361304)
    (0.01665780, 0.02167459)
    (0.01369190, 0.01977039)
    (0.01108039, 0.01788830)
    (0.008785779, 0.01601716)
    (0.006781116, 0.01414711)
    (0.005048412, 0.01227024)
    (0.003577171, 0.01038135)
    (0.002362721, 0.008479204)
    (0.001403676, 0.006567576)
    (0.0006990856, 0.004657032)
    (0.0002435545, 0.002761454)
    (0.00002599979, 0.0009056400)
    (0.00002599979, -0.0009056400)
    (0.0002435545, -0.002761454)
    (0.0006990856, -0.004657032)
    (0.001403676, -0.006567576)
    (0.002362721, -0.008479204)
    (0.003577171, -0.01038135)
    (0.005048412, -0.01227024)
    (0.006781116, -0.01414711)
    (0.008785779, -0.01601716)
    (0.01108039, -0.01788830)
    (0.01369190, -0.01977039)
    (0.01665780, -0.02167459)
    (0.02002796, -0.02361304)
    (0.02386683, -0.02559846)
    (0.02825571, -0.02764358)
    (0.03329449, -0.02976012)
    (0.03910103, -0.03195685)
    (0.04580557, -0.03423655)
    (0.05353650, -0.03659163)
    (0.06239496, -0.03899972)
    (0.07242320, -0.04142167)
    (0.08358167, -0.04380527)
    (0.09575211, -0.04609495)
    (0.1087684, -0.04824299)
    (0.1224568, -0.05021633)
    (0.1366645, -0.05199727)
    (0.1512710, -0.05358012)
    (0.1661868, -0.05496702)
    (0.1813469, -0.05616449)
    (0.1967044, -0.05718119)
    (0.2122249, -0.05802654)
    (0.2278824, -0.05871002)
    (0.2436573, -0.05924080)
    (0.2595337, -0.05962761)
    (0.2754992, -0.05987865)
    (0.2915433, -0.06000157)
    (0.3076573, -0.06000357)
    (0.3238340, -0.05989134)
    (0.3400671, -0.05967113)
    (0.3563512, -0.05934879)
    (0.3726814, -0.05892981)
    (0.3890536, -0.05841929)
    (0.4054640, -0.05782205)
    (0.4219094, -0.05714258)
    (0.4383865, -0.05638509)
    (0.4548929, -0.05555352)
    (0.4714257, -0.05465156)
    (0.4879829, -0.05368266)
    (0.5045622, -0.05265006)
    (0.5211615, -0.05155676)
    (0.5377791, -0.05040555)
    (0.5544131, -0.04919904)
    (0.5710618, -0.04793964)
    (0.5877237, -0.04662956)
    (0.6043972, -0.04527084)
    (0.6210807, -0.04386535)
    (0.6377730, -0.04241478)
    (0.6544725, -0.04092067)
    (0.6711780, -0.03938437)
    (0.6878882, -0.03780710)
    (0.7046016, -0.03618991)
    (0.7213171, -0.03453371)
    (0.7380334, -0.03283924)
    (0.7547492, -0.03110713)
    (0.7714632, -0.02933785)
    (0.7881741, -0.02753174)
    (0.8048802, -0.02568905)
    (0.8215796, -0.02380992)
    (0.8382694, -0.02189448)
    (0.8549446, -0.01994301)
    (0.8715962, -0.01795616)
    (0.8882062, -0.01593565)
    (0.9047380, -0.01388559)
    (0.9211172, -0.01181512)
    (0.9371981, -0.009743313)
    (0.9527169, -0.007706239)
    (0.9672663, -0.005761861)
    (0.9803692, -0.003981369)
    (0.9916796, -0.002421450)
    (1.000000, -0.001260000)
  };
\end{scope}
}

\newcommand\DrawCamberedAirfoil[1][]{
\begin{scope}[#1]
  % NACA 2412
  \draw plot [smooth] coordinates {
    (1.0, -0.0012599999999999777)
    (0.96549018468297, -0.003766550717226144)
    (0.930883045462546, -0.006219202442366447)
    (0.8961437971575432, -0.008631733865730334)
    (0.8613016381510661, -0.011011201185171479)
    (0.8263867428041894, -0.013362379670991803)
    (0.7914310164102236, -0.015687541221114883)
    (0.7564670891854666, -0.01798637927354734)
    (0.7215286223342008, -0.020255875905877893)
    (0.6866502816986508, -0.022490225407487845)
    (0.6518674215422855, -0.02468081055332231)
    (0.6172164265648663, -0.026816170723141183)
    (0.5827346534458342, -0.02888202916719134)
    (0.5484605052100948, -0.03086134225579657)
    (0.5144339818074576, -0.032734349703361115)
    (0.480696823454966, -0.03447867568513195)
    (0.4472934967551756, -0.03606940997738771)
    (0.41393567802086406, -0.037492644839771605)
    (0.3804149998506693, -0.03874755535699173)
    (0.3468272731008735, -0.039902691513794025)
    (0.3136190982592265, -0.04090782505712402)
    (0.2811814440525585, -0.04170125833576889)
    (0.2496338451339312, -0.04222939050878645)
    (0.21912534718162296, -0.04243729854617861)
    (0.1898439100344943, -0.042270895665989155)
    (0.16202614783530983, -0.0416809040510345)
    (0.13596390146008333, -0.04062966635981699)
    (0.11199933361948029, -0.039101582938232377)
    (0.09049621574701124, -0.037116349985295505)
    (0.07361924489626276, -0.0350073364191772)
    (0.05636259357037546, -0.0321514577872086)
    (0.04397649402900053, -0.02948274558741349)
    (0.031214918278790867, -0.025913353166771468)
    (0.025386870661493165, -0.02387293613730815)
    (0.01986852747692683, -0.021588341403196154)
    (0.015810475611440528, -0.019602135923702295)
    (0.012032895834446736, -0.01741586159249026)
    (0.008347783658642402, -0.014804213953756641)
    (0.006782776357419023, -0.013476228004680135)
    (0.005101987955964444, -0.01182609576308366)
    (0.004152048570182651, -0.010748129365261013)
    (0.0030603428943254468, -0.009316573595328595)
    (0.0023446574801605675, -0.008213642891790378)
    (0.001731083959066247, -0.007107510715816521)
    (0.0012037942711423102, -0.005969188316183293)
    (0.0006943488170150654, -0.00457210891942128)
    (0.0002949883265427625, -0.0030077870143021875)
    (0.00013268271827913483, -0.002028658161884189)
    (0.000047353525254730774, -0.0012175353029981746)
    (0.00002532476506966959, 0.0008970844059684918)
    (0.00018814793077252665, 0.002448072322554881)
    (0.00037457810981088356, 0.00345681778994653)
    (0.0006226843506845091, 0.004460305424217867)
    (0.0009189939842633151, 0.0054224385524775705)
    (0.0016420541089524005, 0.007257797625008717)
    (0.0028235547140623278, 0.00953206869809826)
    (0.0039902533393324745, 0.01134485186269217)
    (0.00508347301295101, 0.012816488263046943)
    (0.006234932100785785, 0.014205330332066607)
    (0.008164412430203494, 0.016273130533696094)
    (0.0109096354233833, 0.01883301025054221)
    (0.0149335517829051, 0.022059247763182688)
    (0.02090169719548933, 0.026119219613711435)
    (0.029090969152910873, 0.0308133496989451)
    (0.03936014651201021, 0.03579205910007571)
    (0.05135871826960699, 0.040760579490798525)
    (0.06527912321719298, 0.04572188569496259)
    (0.08128812301403121, 0.05064207626527274)
    (0.09937610494377608, 0.05542524513564439)
    (0.11929438296032827, 0.0599347534152128)
    (0.14075992658866143, 0.06406441454515942)
    (0.1635174601103429, 0.06774419165547137)
    (0.18735476998529443, 0.07093243648306191)
    (0.21210090281011207, 0.07360721154371101)
    (0.23761935376120738, 0.0757595916459023)
    (0.2638006186070965, 0.07738903218729433)
    (0.29055586446557025, 0.07850032882407872)
    (0.31781188174707375, 0.07910164537520417)
    (0.34550708677686826, 0.07920323908566278)
    (0.37365456652101114, 0.07881517686907744)
    (0.402631894340016, 0.07793063620654582)
    (0.43241230840392214, 0.07660997118115524)
    (0.46290547434324336, 0.07490494347854927)
    (0.4936649080396488, 0.07284825058170255)
    (0.5246548772457837, 0.0704561000377626)
    (0.5558421106966199, 0.06774368002361203)
    (0.5871951553016743, 0.06472507690681442)
    (0.6186840935139319, 0.061413195624036104)
    (0.650280234781535, 0.05781970701641605)
    (0.6819560363938716, 0.053954993285283706)
    (0.7136849190124785, 0.049828124742128076)
    (0.7454411587455343, 0.045446849321571406)
    (0.7771999395698983, 0.040817578741059804)
    (0.8089371985433962, 0.0359454208624503)
    (0.840629806069205, 0.030834179949050796)
    (0.8722552320885923, 0.025486452979646447)
    (0.903791999768359, 0.01990361393618675)
    (0.935219453987816, 0.014085919535834577)
    (0.9672766290279255, 0.007882534061003747)
    (1.0, 0.0012599999999999777)
  };
\end{scope}
}

\makeatletter
\dspdeclareoperator{dspvoidshapeadder}{
  % Coordinate offset for the plus
  \pgfutil@tempdima=\radius
  \pgfutil@tempdima=0.55\pgfutil@tempdima
  \pgfusepathqstroke
}
\tikzset{
  vdspadder/.style={
    shape=dspvoidshapeadder,
    line cap=rect,
    line join=rect,
    line width=\dspblocklinewidth,
    minimum size=\dspoperatordiameter,
    label={185:$+$},
    label={265:$-$}
  },
  vadspadder/.style={
    shape=dspvoidshapeadder,
    line cap=rect,
    line join=rect,
    line width=\dspblocklinewidth,
    minimum size=\dspoperatordiameter,
    label=below right:$-$,
    label=above right:$+$
  },
  vkdspadder/.style={
    shape=dspvoidshapeadder,
    line cap=rect,
    line join=rect,
    line width=\dspblocklinewidth,
    minimum size=\dspoperatordiameter,
    label=below right:$-$,
    label=below left:$+$
  },
  vjdspadder/.style={
    shape=dspvoidshapeadder,
    line cap=rect,
    line join=rect,
    line width=\dspblocklinewidth,
    minimum size=\dspoperatordiameter,
    label=above right:$+$,
    label=below right:$-$
  },
  edspadder/.style={
    shape=dspvoidshapeadder,
    line cap=rect,
    line join=rect,
    line width=\dspblocklinewidth,
    minimum size=\dspoperatordiameter
  }
}
\makeatother


\tikzset{dspblock/.style={inner xsep = 5pt, inner ysep = 5pt}}
\tikzset{dspblock2/.style={inner xsep = 5pt, inner ysep = 10pt}}
\tikzset{dspblock3/.style={inner xsep = 5pt, inner ysep = 20pt}}

\newcommand{\DrawCoordinateSystem}[4] {
  \begin{scope}[#1]
    \draw[dashed, ->] (0, 0) -- (-1, 0) node[below] {#2};
    \draw[dashed, ->] (0, 0) -- (0, -1) node[right] {#4};
    \draw (0, 0) node[below right] {#3};
    \begin{scope}[rotate=45, scale=0.1]
      \draw (0, 0) circle (1);
      \draw (-1, 0) -- (1, 0);
      \draw (0, -1) -- (0, 1);
    \end{scope}
  \end{scope}
}

\newcommand{\DrawCoordinateSystemDot}[4] {
  \begin{scope}[#1]
    \draw[dashed, ->] (0, 0) -- (-1, 0) node[below] {#2};
    \draw[dashed, ->] (0, 0) -- (0, -1) node[right] {#4};
    \draw (0, 0) node[below right] {#3};
    \begin{scope}[rotate=45, scale=0.1]
      \draw (0, 0) circle (1);
      \draw[black, fill=black] (0, 0) circle (0.33);
    \end{scope}
  \end{scope}
}

% for k = 1:length(x), fprintf('\n(%f, %f)', x(k), cosh(x(k) - 0.2)), end
\newcommand\DrawCatenary[1][]{
\begin{scope}[#1]
  \draw plot [smooth] coordinates {
    (0.000000, 1.020067)
    (0.010101, 1.018085)
    (0.020202, 1.016207)
    (0.030303, 1.014433)
    (0.040404, 1.012762)
    (0.050505, 1.011195)
    (0.060606, 1.009731)
    (0.070707, 1.008370)
    (0.080808, 1.007112)
    (0.090909, 1.005956)
    (0.101010, 1.004904)
    (0.111111, 1.003953)
    (0.121212, 1.003105)
    (0.131313, 1.002360)
    (0.141414, 1.001717)
    (0.151515, 1.001176)
    (0.161616, 1.000737)
    (0.171717, 1.000400)
    (0.181818, 1.000165)
    (0.191919, 1.000033)
    (0.202020, 1.000002)
    (0.212121, 1.000073)
    (0.222222, 1.000247)
    (0.232323, 1.000522)
    (0.242424, 1.000900)
    (0.252525, 1.001380)
    (0.262626, 1.001962)
    (0.272727, 1.002646)
    (0.282828, 1.003432)
    (0.292929, 1.004321)
    (0.303030, 1.005312)
    (0.313131, 1.006406)
    (0.323232, 1.007603)
    (0.333333, 1.008902)
    (0.343434, 1.010304)
    (0.353535, 1.011810)
    (0.363636, 1.013418)
    (0.373737, 1.015130)
    (0.383838, 1.016946)
    (0.393939, 1.018865)
    (0.404040, 1.020889)
    (0.414141, 1.023016)
    (0.424242, 1.025248)
    (0.434343, 1.027584)
    (0.444444, 1.030026)
    (0.454545, 1.032572)
    (0.464646, 1.035224)
    (0.474747, 1.037981)
    (0.484848, 1.040844)
    (0.494949, 1.043814)
    (0.505051, 1.046890)
    (0.515152, 1.050073)
    (0.525253, 1.053363)
    (0.535354, 1.056760)
    (0.545455, 1.060265)
    (0.555556, 1.063879)
    (0.565657, 1.067601)
    (0.575758, 1.071431)
    (0.585859, 1.075372)
    (0.595960, 1.079422)
    (0.606061, 1.083582)
    (0.616162, 1.087852)
    (0.626263, 1.092234)
    (0.636364, 1.096727)
    (0.646465, 1.101332)
    (0.656566, 1.106049)
    (0.666667, 1.110879)
    (0.676768, 1.115823)
    (0.686869, 1.120880)
    (0.696970, 1.126052)
    (0.707071, 1.131339)
    (0.717172, 1.136741)
    (0.727273, 1.142259)
    (0.737374, 1.147893)
    (0.747475, 1.153645)
    (0.757576, 1.159515)
    (0.767677, 1.165502)
    (0.777778, 1.171609)
    (0.787879, 1.177835)
    (0.797980, 1.184181)
    (0.808081, 1.190649)
    (0.818182, 1.197237)
    (0.828283, 1.203948)
    (0.838384, 1.210782)
    (0.848485, 1.217739)
    (0.858586, 1.224820)
    (0.868687, 1.232027)
    (0.878788, 1.239359)
    (0.888889, 1.246818)
    (0.898990, 1.254403)
    (0.909091, 1.262117)
    (0.919192, 1.269960)
    (0.929293, 1.277932)
    (0.939394, 1.286034)
    (0.949495, 1.294268)
    (0.959596, 1.302634)
    (0.969697, 1.311133)
    (0.979798, 1.319765)
    (0.989899, 1.328532)
    (1.000000, 1.337435)
  };
\end{scope}
}

\newcommand\DrawCenterOfMass[1][]{
  \begin{scope}[#1]
    \fill (0, 0) -- (1, 0) arc [radius=1, start angle=0, end angle=90];
    \fill (0, 0) -- (0, -1) arc [radius=1, start angle=270, end angle=180];
    \draw (0, 0) circle [radius=1];
  \end{scope}
}

\newcommand\DrawWingFrontWithCoordinateSystem[1][]{
  \begin{scope}[#1]
    \DrawCenterOfMass[scale=0.3]
    \draw[dashed, ->] (0, 0) -- (-2, 0) node[above right] {$y_b$};
    \draw[dashed, ->] (0, 0) -- (0, -2) node[above left] {$z_b$};
  \end{scope}
  \DrawWingFront[#1]
}

\newcommand\DrawWingFront[1][]{
  \begin{scope}[#1]
    % Main wing.
    \draw plot [smooth] coordinates {
      (-12.65, 1.45)
      (-12.43, 0.40)
      (-9.5, 0.23)
      (-6.43, 0.07)
      (-3.36, 0.07)
      (-0.3, 0.07)
      (0.3, 0.07)
      (3.36, 0.07)
      (6.43, 0.07)
      (9.5, 0.23)
      (12.43, 0.40)
      (12.65, 1.45)
      (12.85, 1.25)
      (12.43, 0.20)
      (9.5, 0.03)
      (6.43, -0.13)
      (3.36, -0.13)
      (0.3, -0.13)
      (-0.3, -0.13)
      (-3.36, -0.13)
      (-6.43, -0.13)
      (-9.5, 0.03)
      (-12.43, 0.20)
      (-12.85, 1.25)
      (-12.65, 1.45)
    };

    % Horizontal tail.
    \draw (-2.44, 3.55) rectangle (2.47, 3.65);

    % Vertical tail.
    \draw (-0.05, 0.0) rectangle (0.05, 3.6);

    % Pylons.
    \draw (3.55, -1.60) rectangle (3.75, 1.22);
    \draw (1.15, -1.60) rectangle (1.3, 1.22);
    \draw (-1.15, -1.60) rectangle (-1.3, 1.22);
    \draw (-3.55, -1.60) rectangle (-3.75, 1.22);

    % Rotors.
    \draw (-3.64, -1.60) circle [radius=1.1];
    \draw (-1.21, -1.60) circle [radius=1.1];
    \draw (1.21, -1.60) circle [radius=1.1];
    \draw (3.64, -1.60) circle [radius=1.1];
    \draw (-3.64, 1.22) circle [radius=1.03];
    \draw (-1.21, 1.22) circle [radius=1.03];
    \draw (1.21, 1.22) circle [radius=1.03];
    \draw (3.64, 1.22) circle [radius=1.03];
  \end{scope}
}

\newcommand\DrawWingSideWithCoordinateSystem[1][]{
  \begin{scope}[#1]
    \DrawCenterOfMass[scale=0.3]
    \draw[dashed, ->] (0, 0) -- (0, 2) node[above right] {$x_b$};
    \draw[dashed, ->] (0, 0) -- (-2, 0) node[above left] {$z_b$};
  \end{scope}
  \DrawWingSide[#1]
}

\newcommand\DrawWingSide[1][]{
  \begin{scope}[#1]
    % Main wing.
    \DrawCamberedAirfoil[shift={(0, 0.36)}, rotate=-90, scale=1.46]

    % Pylons.
    \draw plot [smooth] coordinates {
      (1.4510, 1.9207)
      (1.1927, 2.3578)
      (0.9825, 1.8961)
      (0.8782, 1.4807)
      (0.0133, 1.3385)
      (-0.0485, 1.3280)
      (-0.4537, 1.2653)
      (-1.2008, 1.1512)
      (-1.3639, 1.5729)
      (-1.6213, 2.0100)
      (-1.8324, 1.5484)
      (-1.8324, 0.1574)
      (-1.6213, 0.1567)
      (-1.3639, 0.1029)
      (-1.2008, 0.0600)
      (-0.4537, -0.4500)
      (-0.0485, -0.4500)
      (0.0133, 0.5645)
      (0.8782, 0.5163)
      (0.9825, 0.5044)
      (1.1927, 0.5054)
      (1.4510, 0.5156)
      (1.4510, 1.9207)
    };

    % Propellers.
    \draw (-1.597 - 1, 1.613) -- (-1.597 + 1, 1.613);
    \draw (1.216 - 1, 1.960) -- (1.216 + 1, 1.960);

    % Fuselage.
    \draw plot [smooth] coordinates {(0, 0) (0, -3) (1.42, -6.7)};

    % H tail.
    \DrawAirfoil[shift={(3.6, -6.25)}, rotate=-90, scale=0.99]

    % V tail.
    \draw (3.45, -6.86) -- (3.45, -7.62) -- (1.42, -7.72) -- (0.0, -7.62) --
          (0.0, -6.89) -- (1.42, -6.70) -- (3.45, -6.86);
  \end{scope}
}

\newcommand\DrawWingBottomWithCoordinateSystem[1][]{
  \begin{scope}[#1]
    \DrawCenterOfMass[scale=0.3]
    \draw[dashed, ->] (0, 0) -- (0, 2) node[above right] {$x_b$};
    \draw[dashed, ->] (0, 0) -- (-2, 0) node[above left] {$y_b$};
  \end{scope}
  \DrawWingBottom[#1]
}

\newcommand\DrawWingBottom[1][]{
  \begin{scope}[#1]
    % Pylons.
    \DrawCamberedAirfoil[shift={(-3.64, 1.32)}, rotate=-90, scale=1.78]
    \DrawCamberedAirfoil[shift={(-1.21, 1.32)}, rotate=-90, scale=1.78]
    \DrawCamberedAirfoil[shift={(1.21, 1.32)}, rotate=-90, scale=1.78]
    \DrawCamberedAirfoil[shift={(3.64, 1.32)}, rotate=-90, scale=1.78]

    % Fuselage.
    \draw (0, 0) -- (0, -6.25);

    % Main wing.
    \draw (-12.85, 0.09) -- (-12.85, -0.58) -- (-9.50, -0.85) --
          (-6.43, -1.1) -- (-0.3, -1.1) -- (0.3, -1.1) -- (6.43, -1.1) --
          (9.50, -0.85) -- (12.85, -0.58) -- (12.85, 0.09) -- (9.50, 0.28) --
          (6.43, 0.36) -- (0.3, 0.36) -- (-0.3, 0.36) -- (-6.43, 0.36) --
          (-9.50, 0.28) -- (-12.85, 0.09);

    % H-Tail.
    \draw (-2.43, -6.68) -- (-2.43, -7.11) -- (2.43, -7.11) --
          (2.43, -6.68) -- (0, -6.25) -- (-2.43, -6.68);

    % V-tail.
    \DrawAirfoil[shift={(0, -6.9)}, rotate=-90, scale=1.03]
  \end{scope}
}


\newcommand{\norm}[1]{\left\lVert#1\right\rVert}

\newcommand{\qhat}{\hat{q}}
\newcommand{\cbar}{\bar{c}}
\newcommand{\qbar}{\bar{q}}
\newcommand{\cmd}{\mathrm{cmd}}
\newcommand{\eff}{\mathrm{eff}}
\newcommand{\wind}{\mathrm{wind}}
\newcommand{\wing}{\mathrm{wing}}
\newcommand{\kite}{\mathrm{kite}}
\newcommand{\nom}{\mathrm{nom}}
\newcommand{\aero}{\mathrm{aero}}
\newcommand{\geom}{\mathrm{geom}}

\begin{document}

\section{Overview}

\begin{equation}
\left[
\begin{array}{cc}
\mathbf{m} & \mathbf{0} \\
\mathbf{0} & \mathbf{I}
\end{array}
\right]
\left[
\begin{array}{c}
\ddot{\vec{x}}_h \\
\dot{\vec{\omega}}
\end{array}
\right]
+
\left[
\begin{array}{c}
\vec{0} \\
\vec{\omega} \times \mathbf{I} \vec{\omega}
\end{array}
\right]
=
\left[
\begin{array}{c}
T C_b^h \hat{x}_b + m \vec{g}_h + \vec{f}^{\mathrm{bridle}}_h \\
\vec{\tau}^{\mathrm{motor}}_b + \vec{\tau}^{\mathrm{bridle}}_b
\end{array}
\right]
\end{equation}

\begin{equation}
C_h^b = \mathbf{1} \cos \theta +
(1 - \cos \theta) \hat{e} \hat{e}^T + \hat{e}_{\times} \sin \theta
\end{equation}

\begin{eqnarray}
m \ddot{x}_h &=& T C_b^h \hat{x}_b - m g + t_{x_h} \\
&=& T ((1 - \cos \theta) e^2_{x_h} + \cos \theta) - m g + t_{x_h} \\
&\approx& T - m g + t_{x_h}
\end{eqnarray}

\begin{eqnarray}
m \ddot{y}_h &=& T C_b^h \hat{x}_b \\
&=& T ((1 - \cos \theta) e_{x_h} e_{y_h} - e_{z_h} \sin \theta) \\
&\approx& -T \theta_z
\end{eqnarray}

\begin{equation}
\mathbf{I}_b \dot{\vec{\omega_b}} = \tau^{\mathrm{motor}}_b
\end{equation}

It's always possible to calculate three SISO controllers in the
principle axis frame and rotate back to the body frame.  The gains
transform as
\begin{equation}
K_b = C_p^b K_p C_b^p
\end{equation}

\begin{equation}
\delta q = q_b^{b'} = (q_h^b)^{-1} \star q_h^{b'}
\end{equation}

\subsection{Hover frame}

\begin{figure}[h]
  \begin{tikzpicture}
    \draw[dashed] (0, 0) -- (2, 0) node[above] {$\theta_p$};
    \draw (0, 0) -- (3, 1) node[midway, above] {$l_p$};
    \draw[dashed] (3, 1) -- (4, 1) node[pos=0.75, above] {$\theta_t$};
    \draw (3, 1) -- (5, 3) node[midway, above] {$l_t$};
    \DrawWingFront[shift={(5, 3)}, scale=0.1, rotate=-45]
    %\DrawCenterOfMass[shift={(5, 3)}, scale=0.1]
    %\node[below right] at (5, 3) {wing};
    \fill[radius=0.05] (0, 0) circle node[above left] {perch};
  \end{tikzpicture}
  \caption{Schematic of the model describing the coupling between the
    wing and the perch.}
\end{figure}


\section{Parameters}

\begin{center}
  \begin{table}[h]
    \begin{tabular}{lllll}
      \hline
      \hline
      Subsystem & Parameter & Symbol & Value & Units \\
      \hline
      Rotor  & Moment-of-inertia & $I_{\mathrm{prop}}$ & 1.1 & kg m$^2$ \\
             & Torque constant & $k_P$ & 0.024 & N-m-s$^2$/rad$^2$ \\
      Wing   & Mass & $m_{\mathrm{wing}}$ & 1422 & kg \\
             & Area & $A$ & 32.9 & m$^2$ \\
             & Span & $b$ & 25.66 & m \\
             & Chord & $c$ & 1.28 & m \\
             & Roll moment-of-inertia & $I_{xx}$ & 30251 & kg m$^2$ \\
             & Pitch moment-of-inertia & $I_{yy}$ & 8074 & kg m$^2$ \\
             & Yaw moment-of-inertia & $I_{zz}$ & 34640 & kg m$^2$ \\
             & Vertical tail area & $A_{vt}$ & 3 & m$^2$ \\
             & Vertical tail position & $r_{vt}$ & [0, 0, 0] & m \\
             & Horizontal tail area & $A_{ht}$ & 3.5 & m$^2$ \\
             & Horizontal tail position & $\vec{r}_{ht}$ & [-6.25, 0, -3.6] & m \\
             & Perch peg position & $r_{\mathrm{peg}}$ & [0, 0, 0] & m \\
             & Drag coefficient side & $C_{D_{\mathrm{side}}}$ & 1.6 & \# \\
             & Area side & $A_{\mathrm{side}}$ & 8.1 & m$^2$ \\
             & Drag coefficient front & $C_{D_{\mathrm{front}}}$ & 2 & \# \\
      Tether & Linear density & $\mu$ & 1 & kg/m \\
             & Diameter & $d_{\mathrm{tether}}$ & 0.023 & m \\
             & Full length & $l_t$ & 430 & m \\

      \hline
      \hline
    \end{tabular}
  \end{table}
\end{center}

\section{Models}

\subsection{Motor model}

The motor model in the simulator is described by the following
equations:
\begin{eqnarray}
I_{\mathrm{prop}} \dot \omega &=&
\tau_{\mathrm{motor}} - \tau_{\mathrm{aero}} \\
\dot \tau_{\mathrm{motor}} &=&
\omega_m (\tau_{\mathrm{cmd}} - \tau_{\mathrm{motor}}) \\
\tau_{\mathrm{cmd}} &=&
k_{{\omega}_p} \left( \omega_{\mathrm{cmd}} - \omega \right) +
k_{{\omega}_i} \int_0^t dt' \left( \omega_{\mathrm{cmd}} - \omega \right) \\
\tau_{\mathrm{aero}} &=& k_P \omega^2
\end{eqnarray}
where $I_{\mathrm{prop}}$ is the moment-of-inertia of the propeller
and rotor, $\omega$ is the angular rate of the motor, $\omega_m$ is a
filter frequency that is meant to simulate the bandwidth of the power
electronics (although this may be misleading as the bandwidth is
better described by the equations below), $k_{\omega}$ is the gain of
the speed control loop, and $k_P$ is the torque constant of the
propellers.  Currently, in the simulator we have $I_{\mathrm{prop}} = 1.1$
kg m$^2$, $k_P = 0.024$ N-m-s$^2$/rad$^2$, $k_{{\omega}_p} = 36.2$ N-m-s/rad,
$k_{{\omega}_i} = 36.2$  N-m-s$^2$/rad.  The typical operating point of the
rotors is around, $\omega_{\mathrm{nom}} = 130$ rad/s.

Based on the motor model equations, the transfer function for the
motor's angular rate is:
\begin{eqnarray}
\frac{\Omega(s)}{\Omega_{\mathrm{cmd}}(s)}
&=& \frac{H(s)}{I_{\mathrm{prop}} s + H(s) + 2 k_P \omega_{\mathrm{nom}}} \\
H(s) &=& \frac{\omega_m}{s + \omega_m} (k_{{\omega}_p} + k_{{\omega}_i} / (s - p))
\end{eqnarray}
The DC gain of the motors is
\begin{equation}
\frac{\Omega(0)}{\Omega_{\mathrm{cmd}}(0)} =
\frac{k_{{\omega}_p} + k_{{\omega}_i} / p}
{k_{{\omega}_p} + k_{{\omega}_i} / p + 2 k_P \omega_{\mathrm{nom}}}
\end{equation}
Here $p = 2 \pi 0.03$ rad/s is a ``fake'' pole that we put into the
simulator so the motor transfer function doesn't fight the stacking
controller.

And thus, we can reasonably describe the overall motor moment, and
thrust, transfer functions by a second order sytem:
\begin{equation}
\frac{M_{\mathrm{motor}}(s)}{M_{\mathrm{cmd}}(s)} =
\frac{\omega_0^2}{s^2 + 2 \zeta \omega_0 s + \omega_0^2}
\end{equation}
For the motors currently in the simulator $\omega_0 = 2 \pi 10$ rad/s and
$\zeta = 1$.

\subsection{Yaw model}

The yaw model is the easiest model to describe.
\begin{equation}
\frac{\Psi(s)}{M_z(s)} = \frac{1}{I_{zz} s^2}
\end{equation}
At the nominal zero yaw rate, aerodynamic damping is effectively zero.

During perching, the yaw model is modified to account for dynamics
from the perch peg.  With the small normal force from the inclined
perch panel and with non-zero peg friction, the system is effectively
an inverted pendulum about the peg.  The transfer function is:
\begin{equation}
\frac{\Psi(s)}{M_z(s)} =
\frac{1}{(I_{zz} + m l_{\mathrm{peg}}^2) s^2 - m g\, l_{\mathrm{peg}}}
\end{equation}
Here $l_{\mathrm{peg}} = 3$ m is the lever arm from the center-of-mass
to the perch peg.

Although aerodynamic forces do not have a large effect on the dynamics
of the yaw model, large tangential velocities, or equivalently, a
large side wind component, results in a significant aerodynamic
moment, which should be eliminated through a feed-forward term:
\begin{equation}
m^{\mathrm{aero}}_z = \frac{1}{2} \rho v_y^2 C_{D_t} A_{vt} (-x_{vt})
\end{equation}
where $C_{D_t} \approx 1.6$ is the vertical tail's drag coefficient,
$A_{vt} = 3$ m$^2$ is the vertical tail's area, and $x_{vt} = -7.2$ m
is the vertical tail's lever arm.  For example, at 90 degrees from
downwind in 10 m/s of wind, the aerodynamic moment would be
approximately 2 kN-m.

Similarly, the tether force, which is applied through an asymmetric
bridle, should be removed through a feed-forward term:
\begin{equation}
m^{\mathrm{tether}}_z = \vec{x}_b \times \vec{t} \approx -t_x\, y_b
\end{equation}
Because the tether force is primarly in the $x$ and $z$ directions,
the effect is mostly due to tether weight.  At full tether extension,
with $t_x \approx 225$ N and $y_b = 0.5$ m, the tether moment would be
113 N-m.

\subsection{Pitch model}

\begin{figure}[h]
  \begin{tikzpicture}
    \begin{scope}[scale=0.4]
      \DrawWingSideWithCoordinateSystem[shift={(20, 0)}]
      \DrawWingBottomWithCoordinateSystem
      \DrawWingFrontWithCoordinateSystem[shift={(0, -13)}]
    \end{scope}
  \end{tikzpicture}
\end{figure}

\begin{figure}[h]
  \begin{tikzpicture}
    \draw[dashed] (0, 0) -- (0, 4) node[below right] {$\theta$};
    %% \draw[thick, ->] (-3, -2.5) -- (-1, -2.5)
    %%                  node[below left] {$\vec{v}_{\mathrm{wind}}$};
    \begin{scope}[shift={(0, 2)}, rotate=-110]
      \DrawCenterOfMass[scale=0.15]
      \DrawCamberedAirfoil[scale=3, shift={(-0.25, 0)}]
      \begin{scope}[shift={(4, 1)}, rotate=20]
        \DrawAirfoil[scale=1.5, shift={(-0.25, 0)}]
        \draw[dashed] (-0.25, 0) -- (-1.5, 0) node[below] {};%{$\alpha_{ht}$};
      \end{scope}

      \draw[thick, ->] (-0.8, 0) -- (-1.8, 0) node[right] {$T$};
      \draw[thick, ->] (1.1, -0.35) -- (2, -2.15) node[above] {$\vec{t}$};

      \draw[dashed] (0, 0) -- (4, 0);
      \draw[dashed] (4, 0) --(4, 1) node [above right] {$(x_{ht}, z_{ht})$};
      \draw[dashed] (3, 1) -- (5, 1);

      \fill[radius=0.05] (1, -0.15) circle node[above left] {$(x_b, z_b)$};

      \draw[->] (5, 1) arc[radius=1, start angle=0, end angle=20]
                       node[right] {$-\delta_e$};
    \end{scope}
  \end{tikzpicture}
  \caption{Side view of a wing showing the forces and variables
    relevant to the pitch model.}
\end{figure}

Away from the perch, the pitch model is similar to the yaw model
except that the tether force couples more strongly to the pitch
dynamics due to the relationship between tension and pitch angle.  For
simplicity, we assume that the tether is perfectly stiff (a poor
assumption at full pay-out).  Then the tether moment is
\begin{equation}
m^{\mathrm{tether}}_y = \vec{x}_b \times \vec{t} \approx -t_z\, x_b
\approx -T \sin \theta x_b
\end{equation}

The elevator aerodynamics are also a significant difference from the
yaw model.  At typical wind speeds for power generation, the elevator
provides significant restoring and damping forces.  The aerodynamic
moment from the elevator is:
\begin{equation}
m^{\mathrm{aero}}_y =
-\frac{1}{2} \rho v_w^2 A_{ht} (-z_{ht}) \frac{dC_L}{d\alpha}
\left(\alpha_0 + \theta + \frac{\dot \theta (-z_{vt})}{v_w}\right)
\end{equation}
With $v_w = 10$ m/s, $A_{ht} = 3.5$ m$^2$, $z_{ht}=-3.6$ m,
$dC_L/d\alpha \approx 4.8$, we find aerodynamic spring and damping
coefficients of $k_{\mathrm{aero}} = 3.6 \times 10^3$ N-m/rad and
$b_{\mathrm{aero}} = 1.3 \times 10^3$ N-m-s/rad.

\begin{equation}
\frac{\Theta(s)}{M_y(s)} =
\frac{1}{I_{yy} s^2 + b_{\mathrm{aero}} s + k_{\mathrm{tether}} + k_{\mathrm{aero}}}
\end{equation}

At large pay-outs, where the tether acts more as a soft spring due to
the catenary, a coupled linear and rotational model is more
appropriate.
\begin{eqnarray}
m \ddot{x} = T \sin \theta - k (x - l_b \sin \theta) \\
I_{yy} \ddot{\theta} = k (x - l_b \sin \theta) l_b + \tau_y
\label{eqn:pitch_system}
\end{eqnarray}
Here $m = 1422$ kg is the wing mass, $l_b = 0.15$ m is the lever arm
from the bridle attachment point to the wing's center-of-mass, $T$ is
the total thrust from the propellers (which is typically significantly
greater than the weight of the wing), and $k$ is the spring constant
due to the tether or catenary.

Linearizing and converting Eq. \ref{eqn:pitch_system} to an equation
in a single variable, we find:
\begin{eqnarray}
I_{yy} \ddddot{\theta} &=&
\frac{k l_b}{m} \left( T \theta - k(x - l_b \theta) - l_b m \ddot{\theta} \right)
+ \ddot{\tau}_y \\
&=&
\frac{k l_b}{m} \left( T \theta + \frac{\tau_y}{l_b}
- \frac{I_{yy}}{l_b} \ddot{\theta}  - l_b m \ddot{\theta} \right) + \ddot{\tau}_y \\
&=&
\frac{k l_b T}{m} \theta + \frac{k}{m} \tau_y
- \left(\frac{k I_{yy}}{m} + k l_b^2 \right) \ddot{\theta} + \ddot{\tau}_y
\end{eqnarray}

\begin{equation}
\left(I_{yy} s^4 + \left(k I_{yy}/m + k l_b^2 \right) s^2
- \frac{k l_b T}{m} \right) \Theta(s) = \left(\frac{k}{m} + s^2\right) M_y(s)
\end{equation}
For large spring constants, close to the perch, this simplifies to:
\begin{equation}
\left((I_{yy} + m l_b^2) s^2 - l_b T \right)  \Theta(s) = M_y(s)
\end{equation}


During launching and perching, when we are in contact with the perch
panel, there is an additional term from the spring constant,
$k_p = 1 \times 10^6$, and lever arm, $l_p = 3$ m, of the perch peg:
\begin{equation}
\left((I_{yy} + m l_b^2) s^2 - l_b T + k_p l_p \right) \Theta(s) = M_y(s)
\end{equation}


\subsection{Bridle model}

The bridle couples the roll, pitch, and yaw models together.  For the
most part, the simple models are sufficient for controller
development.  However, ...


\begin{eqnarray}
\vec{m}_{\mathrm{bridle}}
&=& \vec{r}_b \times C_g^b \vec{t}_g \\
&=& (\vec{r}_b^{\mathrm{fixed}} + \vec{r}_b^{\mathrm{pivot}}) \times C_g^b \vec{t}_g
\end{eqnarray}
where
\begin{equation}
\vec{r}_b^{\mathrm{pivot}} = R_{\mathrm{bridle}}
\frac{C_g^b \hat{t}_g -  (C_g^b \hat{t}_g \cdot \hat{y}_b) \hat{y}_b}
{|C_g^b \hat{t}_g -  (C_g^b \hat{t}_g \cdot \hat{y}_b) \hat{y}_b|}
\end{equation}

Using the identity $\dot{C_g^b} = [\Omega]_{\times} C_g^b$ where
\begin{equation}
[\Omega]_{\times} = \left(
\begin{array}{ccc}
0 & d\psi & -d\theta \\
-d\psi & 0 & d\phi \\
d\theta & -d\phi & 0
\end{array}
\right)
\end{equation}

\begin{eqnarray}
\frac{\partial \vec{m}_b^{\mathrm{bridle}}}{\partial (\phi, \theta, \psi)}
&=&
\vec{r}_b^{\mathrm{fixed}} \times [\Omega]_{\times} \vec{t}_b \\
&+& \frac{R_{\mathrm{bridle}}}
{|\hat{t}_b -  (\hat{t}_b \cdot \hat{y}_b) \hat{y}_b|}
\left(
(-([\Omega]_{\times} \hat{t}_b \cdot \hat{y}_b) \hat{y}_b \times \vec{t}_b
-(\hat{t}_b \cdot \hat{y}_b) \hat{y}_b \times [\Omega]_{\times} \vec{t}_b \right. \\
&+& \left. (\hat{t}_b \cdot \hat{y}_b) \hat{y}_b \times \vec{t}_b
\frac{|[\Omega]_{\times} \hat{t}_b -
([\Omega]_{\times} \hat{t}_b \cdot \hat{y}_b) \hat{y}_b|}{|\hat{t}_b -
(\hat{t}_b \cdot \hat{y}_b) \hat{y}_b|} \right) \\
&+& |dt| \cdot (\vec{r}_b \times \hat{t}_b)
\end{eqnarray}

\begin{equation}
|dt| = T \left(
\begin{array}{c}
0 \\
-d\psi \\
d\theta
\end{array}
\right) \cdot \hat{t}
\end{equation}


\subsection{Altitude model}

Far from the perch, the altitude model is
\begin{equation}
\frac{X(s)}{T(s)} =
\frac{1}{m_{\mathrm{eff}} s^2}
\end{equation}
where $m_{\mathrm{eff}} = m_{\mathrm{wing}} + m_{\mathrm{tether}}/3$.
Near the perch, we include the downward force of the tether.
\begin{equation}
\frac{X(s)}{T(s)} =
\frac{1}{m_{\mathrm{eff}} s^2 + t / l_{\mathrm{tether}}}
\end{equation}
where $t$ is the nominal tension in the tether and
$l_{\mathrm{tether}}$ is the payed tether length.

\subsection{Tangential position model}

\begin{equation}
(m_{\mathrm{wing}} + m_{\mathrm{tether}}/3) \ddot{y}_h = T \sin \psi -
\frac{1}{2} \rho \dot{y}^2 (C_{D_{\mathrm{side}}} A_{\mathrm{side}} +
C_{D_{\mathrm{tether}}} d_{\mathrm{tether}} l_{\mathrm{tether}}/4)
\end{equation}
$C_{D_{\mathrm{side}}} = 1.6$, $A_{\mathrm{side}} = 8.1$ m$^2$ including
the pylon side area and the vertical tail side area.
$C_{D_{\mathrm{tether}}} = 1.2$ and $d_{\mathrm{tether}} = 0.023$ m.

\begin{equation}
\frac{Y(s)}{\Psi(s)} =
\frac{T}{m_{\mathrm{eff}} s^2 + \rho |v_y| C_{D_{\mathrm{eff}}} A_{\mathrm{side}} s}
\end{equation}

\subsubsection{Tangential position with perch model}

\begin{figure}[h]
  \begin{tikzpicture}
    \draw[dashed] (0, 0) -- (2, 0) node[above] {$\theta_p$};
    \draw (0, 0) -- (3, 1) node[midway, above] {$l_p$};
    \draw[dashed] (3, 1) -- (4, 1) node[pos=0.75, above] {$\theta_t$};
    \draw (3, 1) -- (5, 3) node[midway, above] {$l_t$};
    \DrawWingFront[shift={(5, 3)}, scale=0.1, rotate=-45]
    %\DrawCenterOfMass[shift={(5, 3)}, scale=0.1]
    %\node[below right] at (5, 3) {wing};
    \fill[radius=0.05] (0, 0) circle node[above left] {perch};
  \end{tikzpicture}
  \caption{Schematic of the model describing the coupling between the
    wing and the perch.}
\end{figure}
When the wing is near the perch, the perch dynamics couple strongly to
the tangential position loop.  The transfer function between the
tangential postion $y$ and the yaw angle $\psi$ is described by the
system:

\begin{equation}
\dot{\left(
\begin{array}{c}
\theta_p \\
\dot{\theta}_p \\
\theta_t \\
\dot{\theta}_t
\end{array}
\right)}
=
\left(
\begin{array}{cccc}
0 & 1 & 0 & 0 \\
-t l_p/I_p & -b_p & t l_p/I_p & 0 \\
0 & 0 & 0 & 1 \\
t l_p^2/l_t I_p & 0 & -t l_p^2/l_t I_p & 0
\end{array}
\right)
\left(
\begin{array}{c}
\theta_p \\
\dot{\theta}_p \\
\theta_t \\
\dot{\theta}_t
\end{array}
\right)
+
\left(
\begin{array}{c}
0 \\
0 \\
0 \\
T/m l_t
\end{array}
\right)
\psi
\end{equation}

\begin{equation}
y = \left(
\begin{array}{cccc}
l_p & 0 & l_t & 0
\end{array}
\right)
\left(
\begin{array}{c}
\theta_p \\
\dot{\theta}_p \\
\theta_t \\
\dot{\theta}_t
\end{array}
\right)
\end{equation}

\newpage
\subsubsection{Tangential position with swinging tether model}

\begin{figure}[h]
  \begin{tikzpicture}
    \draw[dashed] (-4, 0) -- (4, 0);
    \draw (0, 0) -- (1, -3) node[midway, above right] {$l_t$};
    \draw[dashed] (0, 0) -- (0, -3) node[midway, right] {$\psi_t$};
    \DrawWingBottom[shift={(0, 0)}, scale=0.2, rotate=-5]
    %\DrawCenterOfMass[shift={(5, 3)}, scale=0.1]
    %\node[below right] at (5, 3) {wing};
    \fill[radius=0.05] (0, 0) circle node[above left] {};
    \fill[radius=0.1] (1, -3) circle node[above right] {$m_t$};
  \end{tikzpicture}
  \caption{Schematic of the model describing the coupling between the
    wing and the tether.}
\end{figure}
When the wing is far from the perch, the swinging modes of the tether
interact strongly with the tangential position controller.  For
simplicity, we assume the wing is constrained to be at a fixed
altitude, that the tether is attached at the center-of-mass, and that
the tether can be modeled as a simple point-mass pendulum.  Very
approximately, the 

\begin{eqnarray}
\mathcal{L} &=& \frac{1}{2} m_w \dot{y}^2 +
\frac{1}{2} m_t (\dot{y} - l_t \dot{\psi_t} \cos \psi_t)^2 +
\frac{1}{2} m_t (l_t \dot{\psi_t} \sin \psi_t)^2 +
m_t g l_t \cos \psi_t \\
&\approx& \frac{1}{2} m_w \dot{y}^2 +
\frac{1}{2} m_t (\dot{y} - l_t \dot{\psi_t})^2 +
m_t g l_t \cos \psi_t
\end{eqnarray}

\begin{equation}
P = -\frac{1}{2} b_w \dot{y}^2 - \frac{1}{2} b_t (\dot{y} - l_t \dot{\psi}_t)^2
\end{equation}

\begin{eqnarray}
(m_w + m_t) \ddot{y} &=& m_t l_t \ddot{\psi}_t - (b_w + b_t) \dot{y}
+ b_t l_t \dot{\psi}_t + T \psi_w \\
m_t l_t^2 \ddot{\psi}_t &=& m_t l_t \ddot{y} - m_t g l_t \psi_t +
b_t(\dot{y} - l_t \dot{\psi}_t) l_t
\end{eqnarray}


\begin{eqnarray}
m_w \ddot{y} &=& m_t (l_t \ddot{\psi_t} - \ddot{y}) - b_w \dot{y} + T \psi_w \\
m_t l_t^2 \ddot{\psi_t} &=& m_t l_t \ddot{y} - m_t g l_t \psi_t -
b_t l_t (l_t \dot{\psi}_t - \dot{y})
\end{eqnarray}
Here, $b_w = \rho |v_y| A_p C_D$ is due to the side drag force on the
pylons and vertical tail.  For $v_y = 1$ m/s, $b_w \approx 20$ N-s/m.
Again, for the tether, $b_t = \rho |v_t| A_t C_D \approx 20$ N-s/m,
which is due to the side drag force on the tether.

\begin{equation}
\dot{\left(
\begin{array}{c}
y_w \\
\dot{y}_w \\
\psi_t \\
\dot{\psi}_t
\end{array}
\right)}
=
\left(
\begin{array}{cccc}
0 & 1 & 0 & 0 \\
0 & -\frac{b_w}{m_w} & -\frac{m_t}{m_w}g & 0 \\
0 & 0 & 0 & 1 \\
0 & \left(\frac{b_t}{m_t l_t} - \frac{b_w}{m_w l_t}\right) &
-\left(1 + \frac{m_t}{m_w}\right) \frac{g}{l_t} &
-\frac{b_t}{m_t}
\end{array}
\right)
\left(
\begin{array}{c}
y_w \\
\dot{y}_w \\
\psi_t \\
\dot{\psi}_t
\end{array}
\right)
+
\left(
\begin{array}{c}
0 \\
\frac{T}{m_w} \\
0 \\
\frac{T}{m_w l_t}
\end{array}
\right)
\psi_w
\end{equation}

It is actually pretty hard to make this system unstable with the
tangential position controller.



\subsection{Radial position model}

\begin{equation}
(m_{\mathrm{wing}} + m_{\mathrm{tether}}/3)(-\delta \ddot{z}_h) =
T \sin \delta \theta -
\frac{1}{2} \rho (v^{\mathrm{app}}_z + \delta \dot{z}_h)^2
C_{D_{\mathrm{front}}} A_{\mathrm{front}} - k_{\mathrm{tether}} (-\delta z_h)
\end{equation}

\begin{equation}
\frac{Z(s)}{\Theta(s)} =
\frac{-T}{m_{\mathrm{eff}} s^2 + \rho |v^{\mathrm{app}}_z| C_{D_{\mathrm{front}}} A_{\mathrm{front}} s
+ k_{\mathrm{tether}}}
\end{equation}
For $v_z = 10$ m/s, $C_{D_{\mathrm{front}}} = 2$, $A_{\mathrm{front}} = 32$ m$^2$,
the damping term is approximately 320 N-s/m.

\subsection{Pitch-radial oscillation}




\subsection{Constrained yaw-position model}

During the first hover tests, the wing will be suspended by a
constaint line from a mast.  The constraint line attaches to the wing
at a temporary mount called the {\it probiscus}, which is
approximately 2 m in front of the wing center.  When the wing is at
partial thrust in this situation, an unstable yaw-position mode can
develop.  This unstable mode persists even if the unconstrained yaw
loop is closed.

The system of equations that describe this situation are:
\begin{eqnarray}
m_{\mathrm{wing}} \ddot{y} &=& T \sin \psi -
\frac{t_c}{l_c} (y + l_{\mathrm{prob}} \sin \psi) \\
I_{zz} \ddot{\psi} &=&
-\frac{t_c}{l_c} (y + l_{\mathrm{prob}} \sin \psi) l_{\mathrm{prob}} + m_z
\end{eqnarray}
where $t_c$ is the constraint tension, $l_c$ is the constraint length,
and $l_{\mathrm{prob}}$ is the length of the probiscus.  Typical
values for these variables are $t_c = 7000$ N, $l_c = 5$ m, and
$l_{\mathrm{prob}} = 2$ m.

Linearizing and reexpressing this system in state-space form, we find:
\begin{equation}
  \bold{x} = [y, \dot{y}, \psi, \dot{\psi}]^T
\end{equation}
\begin{equation}
A = \left(
\begin{array}{cccc}
   0 & 1 & 0 & 0 \\
   -t_c/m_{\mathrm{wing}} l_c &
   0 &
   g - t_c (l_c + l_{\mathrm{prob}}) / m_{\mathrm{wing}} l_c &
   0 \\
   0 & 0 & 0 & 1 \\
   -t_c l_{\mathrm{prob}} / I_{zz} l_c &
   0 &
   -t_c l^2_{\mathrm{prob}}/I_{zz} l_c  &
   0
\end{array}
\right)
\end{equation}
\begin{equation}
B = \left(
\begin{array}{c}
  0 \\
  0 \\
  0 \\
  I^{-1}_{zz}
\end{array}
\right)
\end{equation}

\begin{figure}[h]
  \begin{tikzpicture}
    \begin{scope}[scale=0.5]
      \fill[radius=0.1] (4, 8) circle node[above left] {constraint};
      \draw[dashed] (4, 8) -- (4, 5);
      \draw (4, 8) -- (-0.52, 2) node[midway, right] {$l_c$};

      \draw[dashed, ->] (0, 0) -- (0, 4) node[above right] {$y_h$}
                        node[left] {$\psi$};
      \draw[dashed, ->] (0, 0) -- (-4, 0) node[above right] {$x_h$};
      \begin{scope}[rotate=15]
        \fill[radius=0.1] (0, 2) circle node[above left] {$\vec{r}_{\mathrm{prob}}$};
        \DrawWingBottomWithCoordinateSystem
      \end{scope}
    \end{scope}
  \end{tikzpicture}
\end{figure}

This system has poles at
\begin{equation}
p = \pm \sqrt{\frac{a+d}{2}
\left(1 \pm \sqrt{1 - 4 \frac{(ad - bc)}{(a + d)^2}}\right)}
\end{equation}
where
\begin{eqnarray}
a &=& \frac{t_c}{m_{\mathrm{wing}} l_c} \\
b &=& \frac{t_c (l_c + l_{\mathrm{prob}})}{m_{\mathrm{wing}} l_c} - g \\
c &=& \frac{t_c l_{\mathrm{prob}}}{I_{zz} l_c} \\
d &=& \frac{t_c l_{\mathrm{prob}}^2}{I_{zz} l_c}
\end{eqnarray}

This system is still unstable under many conditions if the yaw loop is
closed.

% Generate yaw_position_constraint.pdf with yaw_position_constraint.m.
% Crop with: pdfcrop -margins 10 yaw_position_constraint.pdf
\begin{figure}[h]
\includegraphics[width=\linewidth]{yaw_position_constraint-crop.pdf}
\end{figure}


\subsection{Tether and catenary spring constants}

The spring constant of the tether itself is simply:
\begin{equation}
F = \frac{E A}{l}
\end{equation}
where $E = 150$ N/m$^2$ is the tether's elastic modulus, $A = 4.15 \times 10^{-4}$
m$^2$ is the tether's cross-sectional area, and $l$ is the payed out
tether length.  Near the perch, $l < 10$ m, this spring constant
dominates.

When the wing is far from the perch, the tether catenary acts as a
softer spring.  To determine the catenary spring constant, we first
introduce a few basic facts about catenaries.  The formula for a
catenary with horizontal tension, $T_0$, and weight per unit length,
$\mu$, is
\begin{equation}
y = a \cosh (x / a)\;\; a = T_0 / \mu
\end{equation}
To simplify matters, we assume that the wing is level with the ground
station tether attachment point.  Then the length of the catenary is
\begin{equation}
s = 2 a \sinh \left(\frac{r}{2a}\right)
\label{eqn:cat_length}
\end{equation}
where $r$ is the horizontal distance between the wing and ground
station.  Substituting the length of the tether, $l_T$, for $s$ we can
numerically solve for the horizontal distance as a function of
horizontal tension.  Finally, the catenary spring constant is the
derivative $\frac{dT_0}{dr}$, which can be found implicity from
Eq. \ref{eqn:cat_length}:
\begin{equation}
\frac{dT_0}{dr} = \frac{\mu}{2}
\frac{\cosh(r / 2a)}{(r/2a)\cosh(r/2a) - \sinh(r/2a)}
\end{equation}
For the M600 tether for the minimum tension set-point $T_0 = 3400$ N,
the catenary spring constant is approximately 80 N/m at full tether
length.  However, this increases significantly at short tether lengths
until it surpasses the spring stiffness of the tether at around 10 m
and 6.7e6 N/m.

% Generate tether_spring.pdf with tether_math.m.
% Crop with: pdfcrop -margins 10 tether_spring-crop.pdf
\begin{figure}[h]
\includegraphics[width=5in]{tether_spring-crop.pdf}
\caption{Catenary spring constant as a function of payout.}
\end{figure}

\section{Control}

\subsection{Attitude control}

\begin{center}
\begin{figure}[h]
\begin{tikzpicture}
  \matrix [row sep=4mm, column sep=10mm] {
    \node[dspnodeopen, dsp/label=above] (n00) {$\vec{\theta}_{\mathrm{cmd}}$}; &
    \node[dspfilter] (n01) {$\Delta\vec{\theta} = f(\vec{\theta}_1, \vec{\theta}_2)$}; &
    \node[dspfilter] (n02) {$C_{\mathrm{angles}}$}; &
    \node[dspfilter] (n03) {$G_{\mathrm{motors}}$}; &
    \node[dspfilter] (n04) {$G_{\mathrm{rigid\,body}}$}; &
    \node[dspnodefull] (n05) {}; &
    \node[dspnodeopen, dsp/label=above] (n06) {$\vec{\theta}$}; \\
    \node[coordinate] (n10) {}; &
    \node[coordinate] (n11) {}; &
    \node[coordinate] (n12) {}; &
    \node[coordinate] (n13) {}; &
    \node[coordinate] (n14) {}; &
    \node[coordinate] (n15) {}; &
    \node[coordinate] (n16) {}; & \\
    \node[coordinate] (n20) {}; &
    \node[coordinate] (n21) {}; &
    \node[coordinate] (n22) {}; &
    \node[coordinate] (n23) {}; &
    \node[coordinate] (n25) {}; &
    \node[coordinate] (n26) {}; & \\
  };

  \draw[dspconn] (n00) -- (n01);
  \draw[dspconn] (n01) -- node[above] {$\vec{\theta}_{\mathrm{error}}$}
                 (n02) -- node[above] {$\vec{m}_{\mathrm{cmd}}$}
                 (n03) -- node[above] {$\vec{m}$} (n04) -- (n05) -- (n06);
  \draw[dspconn] (n05) |- (n15) -| (n01);
\end{tikzpicture}
\end{figure}
\end{center}

The attitudes in the hover controller are described in terms of an
axis-angle error vector from a reference hover frame.  The hover frame
is defined such that $x$ points upward, or more exactly along
$-z_{\mathrm{NED}}$, $z$ points toward the ground station but is level
with the ground, or again more exactly in the $x-y$ plane of the NED
system.  And finally, $y$ forms a right-handed coordinate system.

The axis-angle representation of the attitude error is convenient
because it is easy to convert directly to the motor moment commands
that are used to correct the angle error.

The trimmed hover attitude changes as a function of payout.  Near the
perch, when the tether is straight, the angle of the bridles is
determined by the altitude of the wing.  However, far from the perch 

\begin{equation}
\vec{r}_b^{\mathrm{bridle}} + \hat{t}_h +  C_h^b \vec{t}_h
\end{equation}


\subsection{Altitude control}
\begin{center}
\begin{figure}[h]
\begin{tikzpicture}
  \matrix [row sep=4mm, column sep=8mm] {
    \node[dspnodeopen, dsp/label=above] (n00) {$h_{\mathrm{cmd}}$}; &[-4mm]
    \node[vdspadder] (n01) {}; &[-4mm]
    \node[dspfilter] (n02) {$C_{\mathrm{altitude}}$}; &
    \node[dspfilter] (n03) {$G_{\mathrm{motors}}$}; &
    \node[dspfilter] (n04) {$G_{\mathrm{altitude}}$}; &
    \node[dspnodefull] (n05) {}; &
    \node[dspnodeopen, dsp/label=above] (n06) {$h$}; \\
    \node[coordinate] (n10) {}; &
    \node[coordinate] (n11) {}; &
    \node[coordinate] (n12) {}; &
    \node[coordinate] (n13) {}; &
    \node[coordinate] (n14) {}; &
    \node[coordinate] (n15) {}; &
    \node[coordinate] (n16) {}; & \\
  };

  \draw[dspconn] (n00) -- (n01);
  \draw[dspconn] (n01) -- (n02) -- node[above] {$T_{\mathrm{cmd}}$} (n03) --
                 node[above] {$T$} (n04) -- (n05) -- (n06);
  \draw[dspconn] (n05) |- (n15) -| (n01);
\end{tikzpicture}
\end{figure}
\end{center}

% \begin{center}
% \begin{figure}[h]
% \begin{tikzpicture}
%   \matrix [row sep=4mm, column sep=8mm] {
%     \node[dspnodeopen, dsp/label=above] (n00) {$r$}; &[-4mm]
%     \node[vdspadder] (n01) {}; &[-4mm]
%     \node[dspfilter] (n02) {$C(s)$}; &
%     \node[dspfilter] (n03) {$G(s)}$}; &[-4mm]
%     \node[dspnodefull] (n04) {}; &
%     \node[dspnodeopen, dsp/label=above] (n05) {$y$}; \\
%     \node[coordinate] (n10) {}; &
%     \node[coordinate] (n11) {}; &
%     \node[coordinate] (n12) {}; &
%     \node[coordinate] (n13) {}; &
%     \node[coordinate] (n14) {}; &
%     \node[coordinate] (n15) {}; & \\
%   };

%   \draw[dspconn] (n00) -- (n01);
%   \draw[dspconn] (n01) -- (n02) -- node[above] {$u$} (n03) -- (n04) -- (n05);
%   \draw[dspconn] (n04) |- (n14) -| (n01);
% \end{tikzpicture}
% \end{figure}
% \end{center}

\begin{center}
\begin{figure}[h]
\begin{tikzpicture}
  \matrix [row sep=4mm, column sep=8mm] {
    \node[dspnodeopen, dsp/label=above] (n00) {$r$}; &[-4mm]
    \node[vdspadder] (n01) {}; &[-4mm]
    \node[dspfilter] (n02) {$G(s)}$}; &[-4mm]
    \node[dspnodefull] (n03) {}; &
    \node[dspnodeopen, dsp/label=above] (n04) {$y$}; \\
    \node[coordinate] (n10) {}; &
    \node[coordinate] (n11) {}; &
    \node[coordinate] (n12) {}; &
    \node[coordinate] (n13) {}; &
    \node[coordinate] (n14) {}; & \\
  };

  \draw[dspconn] (n00) -- (n01);
  \draw[dspconn] (n01) -- (n02) -- (n03) -- (n04);
  \draw[dspconn] (n03) |- (n13) -| (n01);
\end{tikzpicture}
\end{figure}
\end{center}


\subsection{Tangential position control}

\subsection{Radial control}

Radial position is not directly controlled; rather, there is a payout
command and a tension setpoint.  Together these automatically set a
radial position.  However, at low apparent wind speeds and long tether
lengths, the radial position is typically very underdamped, which can
cause large overshoots and long settling times if the wing is started
from a position far from equilibrium (e.g. after a switch from a
piloted mode).  To address this, a damping term is added to the radial
position loop based on the radial velocity.

The radial damping coefficient $k_d$ is chosen such that the poles
from the closed pitch loop and the poles of the tether spring mode
have the same damping ratio.  Specifically, we maximize the minimum
damping ratio.  At short tether lengths, where the effective spring
constant of the catenary is high or even dominated by the elastic
spring constant, this will no longer work because there will not be
sufficient phase margin.  Thus, the damping term is removed at short
tether lengths.

\subsection{Tension control}

The tension controller attempts to meet a tension command,
$t_{\mathrm{cmd}}$, by adjusting the pitch command,
$\theta_{\mathrm{cmd}}$.  This compensator must work well over a wide
range of system models including the near-perch pitch model, the
payed-out pitch-spring model, and finally, for robustness, when the
system has lost all tension on the tether.  We divide the compensator
into two components.  The $C_{\mathrm{tension}}$ compensator is a
simple PI controller (or sometimes just I controller) on the tension
error.  To avoid taking a derivative of the noisy tension measurement,
we use our estimation of radial velocity to apply the damping term.
This differs from the tension derivative by a scaling of the spring
constant, and thus has the additional advantage of allowing us to
choose a single damping constant that works well across the entire
range of spring constants experienced during payout.  With a
relatively large damping constant, we can use the same tension
controller away from the perch, with the soft-spring system, that we
use near the perch without exciting the radial-pitch spring mode.

\begin{center}
\begin{figure}[h]
\begin{tikzpicture}
  \matrix [row sep=4mm, column sep=8mm] {
    \node[dspnodeopen, dsp/label=above] (n00) {$t_{\mathrm{cmd}}$}; &[-4mm]
    \node[vdspadder] (n01) {}; &[-4mm]
    \node[dspfilter] (n02) {$C_{\mathrm{tension}}$}; &[-8mm]
    \node[vdspadder] (n03) {}; &
    \node[dspfilter] (n05) {$G^{\mathrm{closed}}_{\mathrm{pitch}}$}; &
    \node[dspfilter] (n06) {$G_{\mathrm{radial}}$}; &
    \node[dspnodefull] (n07) {}; &[-4mm]
    \node[dspsquare] (n08) {$k_{\mathrm{teth}}$}; &[-4mm]
    \node[dspnodefull] (n09) {}; &
    \node[dspnodeopen, dsp/label=above] (n0a) {$t$}; \\
    \node[coordinate] (n10) {}; &
    \node[coordinate] (n11) {}; &
    \node[coordinate] (n12) {}; &
    \node[coordinate] (n13) {}; &
    \node[dspsquare] (n15) {$k_d s$}; &
    \node[coordinate] (n16) {}; &
    \node[coordinate] (n17) {}; &
    \node[coordinate] (n18) {}; &
    \node[coordinate] (n19) {}; &
    \node[coordinate] (n1a) {}; & \\
    \node[coordinate] (n20) {}; &
    \node[coordinate] (n21) {}; &
    \node[coordinate] (n22) {}; &
    \node[coordinate] (n23) {}; &
    \node[coordinate] (n25) {}; &
    \node[coordinate] (n26) {}; &
    \node[coordinate] (n27) {}; &
    \node[coordinate] (n28) {}; &
    \node[coordinate] (n29) {}; &
    \node[coordinate] (n2a) {}; & \\
  };

  \draw[dspconn] (n00) -- (n01);
  \draw[dspconn] (n01) -- (n02) -- (n03);
  \draw[dspconn] (n03) -- node[above] {$\theta_{\mathrm{cmd}}$} (n05)
                       -- node[above] {$\theta$} (n06)
                       -- node[above] {$-\delta z$} (n07)
                       -- (n08) -- (n09) -- (n0a);
  \draw[dspconn] (n07) |- (n15) -| (n03);
  \draw[dspconn] (n09) |- (n29) -| (n01);
\end{tikzpicture}
\end{figure}
\end{center}

\begin{center}
\begin{figure}[h]
\begin{tikzpicture}
  \matrix [row sep=4mm, column sep=8mm] {
    \node[dspnodeopen, dsp/label=above] (n00) {$t_{\mathrm{cmd}}$}; &[-4mm]
    \node[vdspadder] (n01) {}; &[-4mm]
    \node[dspfilter] (n02) {$C_{\mathrm{tension}}$}; &[-8mm]
    \node[vdspadder] (n03) {}; &
    \node[dspfilter] (n05) {$G^{\mathrm{closed}}_{\mathrm{pitch}}$}; &
    \node[dspfilter] (n06) {$G_{\mathrm{radial}}$}; &
    \node[dspnodefull] (n07) {}; &[-4mm]
    \node[dspsquare] (n08) {$k_{\mathrm{teth}}$}; &[-4mm]
    \node[dspnodefull] (n09) {}; &
    \node[dspnodeopen, dsp/label=above] (n0a) {$t$}; \\
    \node[coordinate] (n10) {}; &
    \node[coordinate] (n11) {}; &
    \node[coordinate] (n12) {}; &
    \node[coordinate] (n13) {}; &
    \node[dspsquare] (n15) {$k_d s$}; &
    \node[coordinate] (n16) {}; &
    \node[coordinate] (n17) {}; &
    \node[coordinate] (n18) {}; &
    \node[coordinate] (n19) {}; &
    \node[coordinate] (n1a) {}; & \\
    \node[coordinate] (n20) {}; &
    \node[coordinate] (n21) {}; &
    \node[coordinate] (n22) {}; &
    \node[coordinate] (n23) {}; &
    \node[coordinate] (n25) {}; &
    \node[coordinate] (n26) {}; &
    \node[coordinate] (n27) {}; &
    \node[coordinate] (n28) {}; &
    \node[coordinate] (n29) {}; &
    \node[coordinate] (n2a) {}; & \\
  };

  \draw[dspconn] (n00) -- (n01);
  \draw[dspconn] (n01) -- (n02) -- (n03);
  \draw[dspconn] (n03) -- node[above] {$\theta_{\mathrm{cmd}}$} (n05)
                       -- node[above] {$\theta$} (n06)
                       -- node[above] {$-\delta z$} (n07)
                       -- (n08) -- (n09) -- (n0a);
  \draw[dspconn] (n07) |- (n15) -| (n03);
  \draw[dspconn] (n09) |- (n29) -| (n01);
\end{tikzpicture}
\end{figure}
\end{center}

\subsection{Path control}

\begin{center}
\begin{figure}[h]
\begin{tikzpicture}
\pgfmathsetmacro{\a}{4}
\pgfmathsetmacro{\b}{7};
\draw[color=gray!50] plot[shift={(2.5, -{\b * cosh(-2.5 / \b)})}, domain=-2.5:5]
    ({\x}, {\b * cosh(\x / \b)});
\DrawWingSide[color=gray!50, shift={(7.5, {cosh(5 / \b)})},
              scale=0.2, rotate=-10];

\draw plot[shift={(2.5, -{\a * cosh(-2.5 / \a)})}, domain=-2.5:4]
    ({\x}, {\a * cosh(\x / \a)});
\DrawWingSide[shift={(6.5, {cosh(4 / \a)})}, scale=0.2,
    rotate=-10];

\DrawCoordinateSystem{shift={(0, 0)}, scale=0.5} {$x_g$}{$y_g$}{$z_g$};
\draw[line width=1pt, ->] (8, {cosh(5 / \b)}) -- (9, {cosh(5 / \b)})
    node[midway, above] {$t_0$};

%\draw plot[shift={(0.5, 0)}, domain=-0.5:5] ({\x}, {\a * cosh(\x / \a)});

%% \DrawCoordinateSystem{shift={(0, 9.2)}, scale=0.5} {$x_g$}{$y_g$}{$z_g$};
%% \DrawCatenary[shift={(0, -1.02)}, scale=10.0];
%% \DrawWingSide[shift={(10, 12.5)}, scale=0.2, rotate=-10]
\end{tikzpicture}
\end{figure}
\end{center}

\subsubsection{Decoupling tension and path}

Path is the outer most loop in the hover controller, but it is not
obvious that it is possible to cleanly separate path from tension
control.  Tension and path are coupled because for any given position
of the wing, we must supply a horizontal tension to keep the tether
from touching the ground.  Moreover, we would like to avoid supplying
excessive horizontal tension to save power, reduce motor heating, and
leave more thrust available for control.  A final constraint is that
we always want to supply a minimum horizontal tension to ensure that
the tether remains wrapped around the drum and the wing maintains roll
stability.  Here we show that for minimum tensions around 3000 N,
increasing horizontal tension and reducing altitude will not decrease
thrust.  And, thus we can simply set a minimum horizontal tension and
decouple the tension and path loops.

To make the math easy, we model the tether catenary using a simple
parabola:
\begin{equation}
h(r) = a r^2 + b r
\end{equation}
Here the coefficients may be related to the horizontal tension, $t_x$,
the tether linear weight density, $\mu$, and the minimum height of the
tether parabola, $h_{\mathrm{min}}$.
\begin{equation}
a = \frac{\mu}{2 t_x}, b = \sqrt{-4 a h_{\mathrm{min}}}
\end{equation}
Making the approximation that the weight of the tether supported by
the wing is equal to the fraction of the radial distance of the tether
after the tether minimum, $r_{\mathrm{min}} = -b / 2a$, multiplied by the total
tether weight, $W_T$, we find that the thrust, $T$, required to
support the tether and the weight of the wing, $W_W$, is:
\begin{equation}
T = \sqrt{(W_T (1 - r_{\mathrm{min}} / r) + W_W)^2 + t_x^2}
\end{equation}
For some combination of $\mu$, $W_T$, $W_W$, $h_{\mathrm{min}}$, and
$t_x$ it may be advantageous to increase horizontal tension above the
minimum required horizontal tension in order to decrease altitude and
support less tether weight.  To determine where this point occurs, we
calculate the derivative of the thrust with respect to horizontal
tension:
\begin{equation}
\frac{dT}{dt_x} = \frac{1}{2 T}
\left[2 t_x - W_{\mathrm{supp.}} \sqrt{\frac{-2 h_{\mathrm{min}} \mu}{t_x}} \right] 
\end{equation}
Here $W_{\mathrm{supp.}} = W_T (1 - r_{\mathrm{min}} / r) + W_W$ is
the total weight supported by the thrust.  Thus, the point when
$\frac{dT}{dt_x}$ becomes negative, and thus it is advantageous to
increase $t_x$ beyond the minimum, occurs when:
\begin{equation}
t_x < \left( \frac{-W_{\mathrm{supp.}}^2 h_{\mathrm{min}} \mu}{2} \right)^{1/3}
\end{equation}
Inserting some representative numbers, $W_{\mathrm{supp.}} = W_T/2 +
W_W = 17000$ N, $h_{\mathrm{min}} = -10$ m, $\mu = 10$ N/m, we find
that the critical horizontal tension is about $2435$ N.  This is
somewhat below our current minimum horizontal tension, so for most
situations simply setting a mimimum horizontal tension irrespective of
path should suffice.



\subsection{Thrust-moment conversion}

The hover controller outputs a command vector
$\bold{u_{\mathrm{hover}}}$ that contains an overall thrust command,
$T$, which is applied at the wing's center-of-mass, and the three
components, $m_i$, of the moment vector.
\begin{equation}
\bold{u_{\mathrm{hover}}} = [T, m_x, m_y, m_z]^T
\end{equation}
This thrust-moment vector must eventually be converted to the
individual motor thrusts described by the vector:
\begin{equation}
\bold{u_{\mathrm{motors}}} = [T_1, T_2, T_3, T_4, T_5, T_6, T_7, T_8]^T
\end{equation}

\begin{equation}
\bold{u_{\mathrm{motors}}} = [\Omega_1, \Omega_2, \Omega_3, \Omega_4,
                          \Omega_5, \Omega_6, \Omega_7, \Omega_8]^T
\end{equation}

\begin{figure}[h]
  \begin{tikzpicture}
    \begin{scope}[scale=0.4]
      \DrawWingFront
    \end{scope}
  \end{tikzpicture}
\end{figure}



However, we are not free to choose these thrusts arbitrarily.  The
stacking topology of the power electronics requires that each block of
motors consume a similar amount of power.  Thus the mean thrust of the
motors in each block must be equal.  This leaves us with five
degress-of-freedom, and a control vector that consists of a common
mode thrust, $T_c$, and four differential thrusts, $\Delta T_{ij}$:
\begin{equation}
\bold{u_{\mathrm{stacking}}} = [T_c, \Delta T_{15}, \Delta T_{26}, \Delta T_{37},
\Delta T_{48}]^T
\end{equation}


\begin{equation}
\min_{\bold{u}_{\mathrm{stacking}}} \norm{\bold{A}\, \bold{u}_{\mathrm{stacking}}
- \bold{u}_{\mathrm{hover}}}^2
\end{equation}

\begin{equation}
\bold{l} \le \bold{C}\, \bold{u}_{\mathrm{stacking}} \le \bold{u}
\end{equation}


\begin{equation}
\bold{x} = [T_c, \Delta T_{15}, \Delta T_{26}, \Delta T_{37},
\Delta T_{48}]^T
\end{equation}

\begin{equation}
\bold{y} = [T, m_x, m_y, m_z]^T
\end{equation}

\begin{equation}
\min_{\bold{x}} \norm{\bold{A}\, \bold{x} - \bold{y}}^2,
\;\;\bold{l} \le \bold{C}\, \bold{x} \le \bold{u}
\end{equation}


In hover, there is a simple relation between motor thrust and torque
\begin{eqnarray}
T = k_T\, \omega^2 \\
\tau = k_P\, \omega^2
\end{eqnarray}

\begin{equation}
\left(\frac{T}{k_T}\right)^3 = \left(\frac{P}{k_P}\right)^2
\end{equation}

Thus, the torque constraint on each motor can be expressed simply as a
thrust constraint:
\begin{equation}
0 \le T_i \le T_{\mathrm{max}}^{\mathrm{(static)}}
\end{equation}

\begin{equation}
\sum_i k_P \left(\frac{T_i}{k_T}\right)^{3/2} \le P_{\mathrm{max}}
\end{equation}



\begin{equation}
\left[
\begin{array}{c}
T_c \\
\Delta T_{15} \\
\Delta T_{26} \\
\Delta T_{37} \\
\Delta T_{48} \\
\end{array}
\right]
=
\begin{array}{c}
T_c \\
\Delta T_{15} \\
\Delta T_{26} \\
\Delta T_{37} \\
\Delta T_{48} \\
\end{array}
\end{equation}



\end{document}
